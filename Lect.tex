%Entry
\documentclass[a4paper, 12pt]{book} %type of the document: report, book;

%Russian language
\usepackage[T1]{fontenc} %code 
\usepackage[utf8]{inputenc} %code of original text
\usepackage[english, russian] {babel} %localisation and transfers

%Images 
\usepackage{graphicx}
\usepackage{makeidx}
%Math pack
\usepackage{amsmath, amsfonts, amssymb, amsthm, mathtools}

% Header
\author{Чернявский Александр Сергеевич}
\title {Лекции по философии. 8 семестр. Преподаватель: В.И. Коцюба}


\begin{document}
\maketitle
\newpage

\addtocontents{toc}{\protect\setcounter{tocdepth}{0}}
Предмет, который мы будем изучать в этом семетре, -- философия. 

\section*{0. Ознакомительная лекция}
\subsection*{1. О термине и нуждах}
Все Вы наверняка слышали слово "Философия", оно греческое и происходит от двух слов: $\varphi \iota \lambda \varepsilon \omega $ -- "любить, иметь приязнь к чему-либо" и
 $\sigma o \varphi \iota \alpha$ -- "мудрость".
 
Первый текст в истории, где встречатеся этот  термин, -- трактат по медицине из Гиппократовского корпуса, датируемый примерно 450-420 гг. до н.э. Этот текст говорит о том, что в 5-ом веке до н.э. термин "философия" уже был привычным в греческой культуре, чего не нужно было пояснять. Из контекста его применения становится понятно, что философы изучали, что такое человек, писали что-то о природе. 

Обратимся ко второй части слова "философия": $\sigma o \varphi \iota \alpha$. Как же понимали эту "софию"? Если мы посмотрим на самую древнюю эпоху (11-9 века до н.э. -- "тёмные века" истории Древней Греции), когда были написаны произвдения Гомера. У него мудрость понимается как "сноровка, умение что-то делать". Но уже в более позднюю эпоху под мудростью начинают понимать и любознательность в широком смысле, и жизненный опыт.

Далее в греческой традиции возникает термин "Пифагорейство". Древние Греки считали, что понятие "философии" ввёл Пифагор, который утверждал, что мудр только Бог, а человек может считаться только любителем мудрости, потому что человек не может изострить свой ум настолько, чтобы считать себя мудрым, стать Богом. 

К настоящему времени философия изучается как некая дисциплина, но что же она изучает, каков её предмет? В разные времена под мудростью понимали разные вещи, была некая неоднозначность того, что значит любить мудрость. Обратившись к первым текстом, с мудростью было связано не только умение познать мир, но также и умение, как жить в правильно в этом мире; философия неразрывно связана с этикой. Впоследствие, во времена Платона и Аристотеля, философия приобретет конкретное определение (Аристотель): предмет философии в том, что она изучает сущее в целом. Если другие науки изучают какие конкретные аспекты реальности, то задача философии как науки -- познать реальность в целом, в общем, как человек познаёт реальность. Также, с точки зрения Аристотеля, важно познать, от чего зависит реальность ("начала реальности"), познать в чём заключается \textit{благо} -- какова конечная цель наших действий (позже: в чём смысл жизни). Те или иные знания -- это средства для достижения смысла жизни. 

В одном из "Диалогов" Платона описывается, как Сократ встретился к Алкивиадом. Сократ, узнав, что Алкивиад пошёл просить у Богов исполнение своего желания, задался вопросом, принесёт ли пользу такая просьба (Алкивиад не задумывался об этом). Ведь человек может достичь желаемого и пожалеть об этом -- если чего-то желаешь, надо задуматься, чтобы не жалеть об исполнении. Существует ли что-то такое в жизни, что исполнится и об этом не пожалеешь? Сократ привёл такой жизненный пример человека, слишком жаждавшего власти, но убитого своими соракниками, после того, как достиг желаемого. Что же нашёл Сократ? Алкивиад решил, что это знание -- особый вид богатства. Знаний -- это благо. Но какое из знаний является наиболее важным для человека? С точки зрения Сократа, знания -- это средства достижения наших целей (всего лишь). Если в порыве гнева один человек исхитрится и помощью имеющихся знаний убьёт другого, то он будет об этом жалеть, что нарушает ценность блага. Отсюда вывод: \textit{знания полезны только тогда, когда мы их направляем для достижения правильных целей}. Самое важное знание, с точки зрения Сократа, -- это умение различать добро и зло; ошибочность в этом вопросе привлекает различные проблемы в нашу жизнь.

Рассмотрим историю философия, её параллели с развитием человеческой истории.

С точки зрения лектора, предметом философии в современном понимании является мировоззрение в широком смысле, система тех норм, ориентиров, идеалов, которыми человек руководствуется по жизни. Каждый философ ставит перед собой свою систему ценностей. Сократ стремился сформировать то, что мы сейчас называем \textit{объективным знанием}, претендуя на универсальность, обосновывая свою позицию. Платон говорил: "Философ -- тот, кто любит истину". Филосфия аккумуилирует опыт и знания, умение жить, выстраивать её распорядок. Также жизнь формируется традицией, например, образовательной -- опыт, полученный философией, получается полезным и в современной жизни.

\subsection*{2. Терминология разделов философии}
\begin{itemize}
\item Онтология.
Этот термин происходит от греческого $o \nu$ -- "сущее". Этот раздел даёт учение о реальности в целом, как она устроена. Философия рассматривает реальность, включая того, кто её познаёт.
 
В философии пригодилось понятие абсолюта. Absolutu (лат.) -- "от всего оторвано, ни от чего не зависит". Всё другое зависит от него, абсолют зависит от самого себя и  только. Философы рассуждают о том, что является тем, от чего зависит всё остальное в мире.

\item Гносеология. Образовано от греческого слова "знание", этот раздел изучает источники, формы и критерии достоверного знания. Началось всё с античности. 

\item Этика. Образовано от греческого слова $\eta \theta o \zeta$ -- "нрав, характер, образ жизни". Этика призвана дать критерии различения добра и зла и правила жизни, как достичь блага.

\item Антропология. Философская антропология (учитывая знания биологии) изучает место человека в реальности и человеческую природу. Homo sapiens -- человек разумный, но в чём проявлется разумность, которая диктует цели, в том числе и достижение совершенства, но что оно из себя представляет?

\item Эстетика. Наука о прекрасном, рассматривает вопрос о критериях красоты, ведь красота зачастую неявно рассматривается как совершенство: ничего ни добавить, ни отнять. Анализ красоты как проявления совершенства, анализа художественного творчества -- предметы эстетики. Художественное творчество -- это выражение мировоззрения, с точки зрения эстетики. 	

Среди "маленьких трагедий" А.С.Пушкина есть очень интересный сюжет "Моцарт и Сальери" о гении и посредственности. Но не всё так гладко! "Вот, говорят, нет правды на Земле,  но правды нет и выше", -- возмущается Сальери. Сальери возмущается тем, что он сделал всю свою жизнь "подножием искусству", но Моцарт хочет жить и простой жизнью, что по мнению Сальери, есть предательство. Вот он всё и исправил. Проблема гениальности и простого служения деньгам -- большая в творчестве Пушкина. Таким образом, художественные произвдения выражают мировоззрение автора.

Ещё один пример -- что выражает человека объективнее, фотография или портрет. Художник рисует человека, как он его видит, каким хочет его видеть, а фотография не предлагает такого участия. Но, с другой стороны, фотография фиксирует миг, мгновенное неинвариантное состояние человека. Портрет объективнее в этом смысле, что на нём изображается внутренний мир человека через внутренний мир художника. Понимание человека проходит через понимание его внутреннего мира. Недаром у Аристотеля поэзия ближе к философии, чем история, потому что история фиксирует единичное событие, а поэзия -- некий закон, правило реальности.

\item Социальная философия. Она ставит вопрос, каким должно быть общество, какие нормы должны быть в жизни, а социология делает выводы о том, каким оно является, отображая картину реальности. 

\item Философия истории, которая рассматривает тему исторического процесса, его направленность. В 18-м веке философы разработали идея прогресса, делая аналогию истории с взрослением человека. Она ставит вопрос о смысле жизни человека, его перспективах.
 	 
\end{itemize}
\textit{Философия ставит цель формирования цельной картины жизни, интегрирующей функции знания.}

\subsection*{3. Античная философия}

Прблема Нидема -- английский учёный, изучавший китайскую культуру, сформулировал такой вопрос: "Почему Китай, имевший высокое культурное развитие, не получил толчка промышленности, как в Европе?" 

Почему античная философия по большей части греческая? По мнению лектора, это может быть связано с тем, что в мировоззрении древних греков религия имела мало нравственных компонентов формирования мировоззрения, в отличие от того же древнего Египта. Боги -- не образцы для людей. Потребность в таком мировоззрении, потребность в таком знании послужила предпосылками к возникновению нормативной философии, дистанцирующей себя от религии.

\subsubsection*{Периодизация греческой философии}
\begin{itemize}
\item[*] Фалес, 6-ой -- 5-ый вв. до н.э.-- период "досократики". Общая черта этого периода: философы интересовались, как устроена природа, откуда всё пошло, поэтому их называют ещё "натурфилософы". По их мнению, "разум призван руководить"

\item[*] Сократ, 5-ый -- 4-ый вв. до н. э.  -- "Классический" (образцовый) период. Появляются 2 очень значимых философа: Платон и Аристотель (по преемственности). Дальнейшая философия -- это некая разработка вопросов, сформулированных произведениями Платона. Аристотель -- это "отец" почти всей науки: физики, этики и т.п. Эти школы пройдут через столетия.

\item[*] 330 -- 30 гг. до н. э. -- Эллинизм -- эпоха, когда благодаря завоеваниям Александра Македонского греческая культура распространилась на огромные территории, греческий язык стал фактически международным, греческая культрура стала базисом взаимодействия других культур. Символ этого периода -- город Александрия.

\item[*] 1-ый век до н.э. -- 6-ый век н.э. -- Позднеантичная, или греко-римская философия. Наступает расцвет комментаторской традиции: если ты умный, начинай понимать и комментировать тексты тех, кто писал до тебя. Типичная техника обучения в те времена: слушали и комментировали прочитанное. 
\end{itemize}
 
\section*{1. Лекция о первом периоде античной философии -- периоде "досократиков"}

\subsection*{1. Этика}
Первым философам-досократикам предшествовали т.н. <<7 мудрецов>>. Их некоторые высказывая сохранились. Основные идеи:

\begin{itemize}
\item Принцип меры 
Один из семи мудрецов, Хилон, выразил этот принцип лаконично: <<Ничего слишком>>. Человеку нужно искать в своих желаниях, поступках и интересах некую меру.

Этот принцип напрямую был применим по потношению к удовольствию. Другой из мудрецов, Клеобул, говорил так: <<Удовольствие обуздывай>>. Или, как говорил, Солон, уважаемый мыслитель, законодатель в Афинах: "Избегай удовольствия, рождающего страдания." Греки обратили внимание на тот факт, что человек, не расчитавший свою скорость, может промахнуться, оступиться -- так и чрезмерное удовольствие рождает старадание.

Мера также проявляет себя через язык, общение с другими людьми. Тот же Клеобул говорил: <<Будь сдержан на язык>>. Цитата Хилона: <<Язык твой пусть не обгоняет ума>>, -- в общении нужна мера, нужно думать, что мы говорим, у каждого разговора должна быть своя разумная цель. Известный поэт Биант Приенский, когда город в котором он жил, претерпел катастрофу, вышел без имущества и объяснил свой поступок фразой: "Всё своё ношу с собой", -- то есть весь мой ум, характер и качества уже и так принадлежат мне. По поводу примера с промахом он и сказал: "Не болтай! Промахнёшься -- пожалеeшь", -- часто в эмоциях человек жалеет о том, что сказал. Мудрость видели в том, что человек умеет следить за своей речью, не сказать лишнего.

Умение управлять собой, обуздать свой гнев также является признаком меры. Солон произнёс: "Научившись подчиняться, научишься и управлять", -- хороший управленец (в том числе и собой) вырастает в умении подчинаяться.

\item Отношения с другими людьми. Общий принцип этих отношений выразил один из 7 мудрецов, Питтак, в <<золотом правиле морали>>, говорящем "что возмущает тебя в другом, того не делай сам". Это правило требует не только того, чтобы следить за другими, но и за своими поступками, речью, делами, -- так мы лучше понимаем принципы отношения и с другими людьми.

Фалес поднял вопрос об \textit{отношении к родителям}: "Какие услуги окажешь родителям, такие в старости ожидают детей".

Вопрос \textit{дружбы} затронул Периандр: <<С друзьями будь одним и тем же, и в удачу, и в беде>>. Как говорил Хилон: "На обеды друзей ходи медленно, а на беды ходи быстро", -- дружба проявлется в том, что оказывается взаимопомощь в беде. Тот же Хилон говорил: "Над попаввшими в беду не смейся".

"Знай себя' , (Хилон) -- из этой фразы следует, что представление человека о себе могут быть ошибочными; античные философы утверждали, что для того чтобы познать себя, необходим первичный самонализ. С точки зрения Фалеса это очень сложно, это кажется очень простым.

\end{itemize}

\subsection*{1. Онтология}

Греческая философия начинается с вопроса, что есть всё, поиска первоэлемента.

Фалес (640 - 546 гг. до н. э.) стал родоначальником Милетской школы. Милет -- город, расположенный на Ионийском побережье Малой Азии. Там появилась первая общепризнанная школа философии.

Примечение: Греки выстраивали свою хронологию по олимпиадам. Период жизни человека обозначался высшей точкой его жизни ($\alpha \kappa \mu \eta$), приходившейся на какую-либо олимпиаду. В древнем Риме на могилах писали эпитафии -- карьерный путь человека.

Фалес считал первоэлементом воду (всё в мире пропитано водой), его ученик Анаксимандр (610 -540 гг. до н.э.) считал первоначалом что-то другое, до сих пор спорят что, но осталась его характеристика, что это начало беспредельное ($\alpha \pi \varepsilon \iota \rho \nu$ ). Вещи на время получают от этого начала свои свойства, но через некоторое время долги нужно вернуть.

Некая аксиома античной онтологии: "Nihil ex nihilo", -- из ничего ничего не бывает. Первые досократики стали различать мир являний, в котором происходит возникнование и исчезновение, за чем существует что-то вечное, что всегда будет. То, что мы видим, -- есть распад и воссоединение этого самого. Это начало можно назвать вечным, ведь увядание красоты вызывало у греков депрессию.

Анаксимандр был одинм из первых прозаиков. Одно из его произвденений -- "О природе" ($\iota \pi \varepsilon \rho \iota  $ $ \varphi \upsilon \sigma \varepsilon \omega \varsigma$). В слове "природа" корень, как и в русском языке, -- "рождать".

Ещё представителем Милетской школы был Анаксимен, который началом всего полагал воздух, он считал, что различные состояния вещества -- это лишь модификации воздуха. 

Далее центр философской жизни перемещается в южную Италию, на Сицилию, где появился центр Пифагорейского движения -- основателем был Пифагор(570 - 497 гг. до н.э.). У учеников этой школы была традиция, они приписывали основателю школы свои произведения.

Пифагорейская школа ищет начало. Она предствляет собой ярко выраженный дуализм, мир объясняется с помощью друх начал, предельного и беспредельного. Предполагают, что Пифагор ввёл термин "космос", от понятия упорядоченной красоты, как противопроставления хаосу. С точки зрения пифагорейцев, космос есть результат взаимодействия этих двух начал; порядок в космосе от предельного начала, которое устанавливает меру, и это красиво. Упорядоченную реальность можно познать. На вопрос, что есть всё, они отвечают: "Всё есть число. Без числа ничто не может быть, без числа ничто нельзя познать". (Филолай). В основе реальности, в которой мы живём, лежат числовые закономерности, то есть число и основа познания реальности.

Пифагорейский союз исследовал чисел, что давало, по их мнению, правильное понимание устройства жизни; нужно находиться в соответствии с миром. Например, в области педагогики они ввели понятие инфантилизма -- отставание умственного развития от телесного -- развитие должно быть гармоничным. Они вводят правило отчёта в конце дня, что в другой день не поддаваться тем же ошибкам, заблуждениям. Пифагорейская школа стала неким образцом для других философских школ.

Среди Досократиков особое место занимает Гераклит (540 - 480 гг. до н.э. плачущий философ). Он плакал, потому что ему было жалко людей, что они зря тратят свою жизнь. Он считал, что люди живут в своих грёзах, когда рассудок у вскх общий. Задача философа -- приобрести этот самый рассудок, "логос" мира (в мире есть некий порядок, некая периодичность), перейти от мнимого к истинному. Ему принадлежит высказывание <<$\pi \alpha \nu \tau \alpha$ $\rho \varepsilon \iota$>>, которое можно перевести как "Всё течёт", изменяется согласно порядку; также мир проникнут некой борьбой, взаимодействием противоложностей; он сравнивал мир с натянутым луком. Первоэлементом Гераклит считал огонь -- Вселенная существует циклически, в определённые моменты рождаясь и сгорая заново; мудрость состоит в том, чтобы познать за многообразием некий общий закон, которому должны подчиняться и люди. Гераклит постоянно критиковал некоторые греческие обычаи, например, пьянство -- поклонение Вакху --, которые были бы нарушением этого закона. Гераклит также известен фразой "Многознание уму не научает", то есть он разделяет знания и мудрость; также он критиковал демократию, считая, что один не тьма, если он наилучший, -- демократия ориентируется на желания толпы, отвергая лучших.
 
Следующей школой греческой философии, которая нам интересна, -- Элейская школа. Она располоагалась в Великой Греции, как и пифагорейская. Первым представителем был Ксенофан -- поэт, его высшая точка жизни -- 540 год до н.э. Выступал с критикой греческой мифологии, критиковал поведение Богов, писал, что Бог един. Ему принадлежит идеология пантеизма, т.е. "всё есть Бог", по учению Ксенофана в мире есть божество, как душа в человеке, оно управляет мыслью и непохоже на людей. Ещё одна мысль, что люди довольствуются только догадками о реальности, а реальная истина всёё же скрыта. 

Ксенофан повлиял на главного представителя Элейской -- Парменида(акмэ -- 504 г до н.э.), написавшего поэму <<о природе>> -- первого последовательного рационалиста. Парменид в своей поэме говорит о том, что к нему снизошло некое откровение: <<Бытие есть, небытия нет>>, ну, и <<Быть и мыслить -- это одно и то же>>, -- совпадение бытия и мышления. Парменид вводит в в философию понятие бытия; всё дело в глаголе "быть", который  в русском языке выступает как глагол-связка, он введён, чтобы отличать то, что есть независимо от наших ощущений, от того, что нам кажется: <<Глаза и уши -- плохие свидетели>> (чувства обманчивы). Реальность такой, какая она есть, открывается нам с помощью мышления. 

Рационализм -- это мировоззрение, которое доверяет только выводам разума, реальность открывется человеку только при помощи мышления. Парменид направил греческую реальность по другому пути, по пути науки. То, что кажется, -- формирует мнение, а то что есть, -- знания.

Следствия из аксиомы Парменида: небытья нет, то бытье вечно, всегда одно и то же, оно совершенно, никуда не может пропасть -- как в законах сохранения; движение -- это тоже иллюзия, с точки зрения Парменида, -- это крайний рационализм. Это привело к некой полемике. Об этом потом писал А.С. Пушкин: "<<Движенья нет>>, -- сказал мудрец; Другой же стал вокруг него ходить". Но чувства обманчивы, как и ощущение движения Солнца вокруг Земли.

Зенон Элейский, ученик Парменида, является отцом диалектики (искусства убеждения). Он предложил несколькой <<апорий>> (проход между рифами, безвыходная ситуация), родоначальников метода доказательств от противного (нельзя пройти бесконечное расстояние за конечное время, например). Другая "апория" Зенона -- про стрелу, основанная на неделимости момента времни (стрела при движении может занимать места больше, чем она есть).

После Элейской школы был Эмпедокл, по его мнению было 4 первоэлемента: вода, огонь, земля и воздух. Есть две силы: любовь и вражда, которые всё рождают и убивают, но количество частичек неизменно.

Первого из Афинских философов -- Анаксагора -- волновал вопрос метаморфозов в природе. Он хотел разобраться в процессе появления чего-то из другого, непохожего. Как потом говорил Аристотель: "Любая философия начинается с умения удивляться". Анаксагор говорил, что "Всё сложно" в том смысле, что во всё есть другие подобочастные частички, и свйоства макрообъекта отличаются количеством микрообъектов -- всё делимо до бесконечности, перераспределением этих частичек и объясняются метаморфозы. Анаксагор вводит понятие мирового ума ($\nu o \upsilon \varsigma $).

 А какую функцию он выполняет, мы узнаем на следующей лекции.

\newpage

\section*{2. Лекция о Сократе и софистах}

На прошлой лекции мы остановились на Анаксагоре и его тезисах о том, что все вещи состоят из мельчайших частиц, которые тоже состоят из мельчайших частик, и так до бесконечности. По представлениям Анаксагора, было некое однородное смешанное вещество из этих качественно неразличимых частиц. Анаксагор вводит понятие мирового космического ума ($\nu o \upsilon \varsigma $), который всё пронизывает и приводит эти мельчайшие частицы в движение, в результате которого, тяжёлые частицы начали собираться в центре, а лёгкие  -- на периферии, так родился космос -- порядок. Эта мысль потом встретится у многих философов: мировой ум как устроитель мира знает о нём всё, а человек всего лишь причастен к устройству реальности.

Анаксагор имел проблемы в Афинах, назвав Солнце <<раскалённой глыбой>> (тогда было принято, что Солнце -- это Божество). Перикл спас его от судебных преследований. 

\subsubsection*{1. Демокрит}
Последним в ряду доскоратиков считается младший современник Анаксагора -- Демокрит (470 -- 360 гг. до н.э.). Он в юности путешествовал, учился и оставил после себя более 50 произведений всякого разного рода: о природе, эстетические, сочинения об ощущении, обонянии, о числах и проч. Но в историю Демокрит вошёл в качестве основателя философского атомизма. Согласно учению Демокрита, сущее неделимо, состоит из неделимых $ \alpha \tau o \mu \varsigma $ -- $ \iota \delta \varepsilon \alpha $. Как буквы, элементарные единицы текста, отличаются формой, так и атомы, <<кирпичики>> всего сущего. Он считал, что атомы могут отличаться формой, положением и последовательностью. Демокрит тоже был рационалистом, но признавал также и $\tau o $ $\mu \eta $ $o \nu$ -- $\tau o $ $\kappa \varepsilon \nu o \nu$ (пустота).

Как и у Анаксагора, у Демокрита присутствует некоторый думализм сущего и пустоты. Как потом напишет Аристотель, Демокрит ввёл это понятие пустоты, потому что считал движение реальным,  а не иллюзорным, как выходило у Парменида. Атомы Демокрита имеют характеристики бытия Парменида, но условие их движения есть пустота.

Демокрит также оставил свой след в истории античной этики: правило жизни, искусство жизни не отдeлимы от познания и умственной деятельности. Слова -- тень деятельности, по его словам. В этике Демокрит делает центральным термин $\varepsilon o \vartheta \upsilon \mu \iota \eta$ (эвтюмия), который можно перевести, как <<хорошее расположение души>>. Он рассуждает, что такое счастье. [его современники считали счастьем удачное совпадение обстоятельств или нечто, дарованное Богами]. Демокрит же считал, что счастье и несчастье в душе; он приводил пример людей, которые завидуют богатым и боятся быть будными, но замечает, что бедный -- это не тот, которому мало, а тот, кто постоянно возмущается тем, что хочется больше. Если человек доволен тем, что у него есть, его эмоциональное состояние становится гораздо лучше. Сам же философ писал: <<Ни телесные силы, ни деньги делают людей счастливыми, а правота и многостронняя их мудрость>>. Человека отличают не телесное развитие, а благородные поступки. Демокрит подчёркивал, что человек должен стыдиться не только других, но и своей совести, выдвигая в качестве идеала человека понятие <<апатия>> (можно перевести как <<бесстрастие>>, <<бесстрадательность>>) -- человек остаётся невозмутимым к внешним раздражителям, не выходя из пределов собственной меры. <<Сильный для него тот, кто сильнее врагов, а тот, кто сильнее своих удовольствий>>, -- пишет философ в работе о храбрости. [В дальнейшем философом называли людей, умеющих преодолевать тяжёлые обстоятельства жизни, не выходя из колеи].

\subsubsection*{2. Новое время античной философии. Философия софистов}

С Демокритом заканчивается период <<Досократиков>>, наступает период 5 -- 4 вв. до н.э. Молодёжь, в том числе и сельская, устремляется в города, для того чтобы достичь каких-то успехов в жизни; меняется привычный до этого уклад. Сама жизнь в полисе, пёстрая, шумная, сделала востребованной образовательную деятельность, чтобы занимать руководящие места, выступать на собраниях.

Речь идёт об образовании, позволяющем влиять на окружающих с помощью слова. Появились софисты -- люди, которые за большие деньги предлагали следующее: в городах устраивали публичные выступления, побеждая любого человека в споре, доказывая и опровергая что угодно. Софисты считали эту науку, риторику, самым важным знанием.

Но какова их философия? 

Наиболее известный из софистов, Протагор, выдвинул философскую теорию, во многом обосновывающую движение софистов. Ему принадлежит фраза: <<Человек -- мера всех вещей>>. Если до этого искали меру как закон существования в реальности, то Протагор утверждает, что каждый сам себе мера, то есть не надо приводить себя в соответствие с чем-то другим, потому что нет что-то такого, что <<существует за мнениямии>>: когда мы ощущаем вещь, мы с ней сталкиваемся, меняя и вещь и собственные ощущения, поэтому образ этой вещи, которую мы имеем в нашых чувствах, уже устарел, потому что ничего не остаётся одним и тем же. Наши образы, слова и ощущения ничему вообще не соответствуют, каждый по-своему прав.
Отсюда, надо забоиться не о том, чтобы наши слова чему-то соответствовали, а чтобы они были сильными, чтобы они влияли на других. Слова меняют людей, по мнению софистов.

Как это отразилось в этике? Сама позиция Протагора и других софистов -- скептицизм. У каждого своё добро и зло. Но как же тогда жить в обществе? Законы общества основаны на какой-то реальности или какая-то удобная договорённость между людьми. Они, конечно, придерживались второй позиции: эти законы можно менять, можно даже им не следовать.

Данный поход, разумеется, приводил к тому, что человек терял <<опору>>, ведь ни на что полагаться более не нужно, только на самого себя. Философия Сократа возникла как раз в этом контексте.  


\subsubsection*{3. Сократ}

Сократ(469 -- 399 гг. до н.э.) же выдвигает своё учение как альтернативу софизму. Он родился и вырос в Афинах, был сыном каменотёса. Заинтересовался филсофией в юности, слушал Анаксагора. Сократ не написал никаких книг, но оставил огромное наследие в виде таких учеников, как Платон (делал Сократа героем своих произвдений), Ксенофонт(написал историю Сократа). Сократ был первым, кто спустил философию с небес на землю, привёл её в дома людей. Позиция Сократа контрастирует с позицией софистов в области различия добра и зла. Для Сократа мудрость -- умение различать добро и зло. 

Вопрос об истине. Он, как и досократики, различает истину и ложь: знание, которое отражает истину, и мнение. Он вырабатывает специальный метод получение знаний, который получил название маевтика (<<повивальное искусство>>): в отличие от софистов, он помогает найти человеку верный путь к знанию и добру. Найти такой путь можно с помощью вопросов и ответов. Сократ начал с определений, выражения сущностей. В качесте примера можно привести встречу философа со своими однополчанами, которые вели своих сыновей к софистам, чтобы обучить их воинскому делу. Завязывается разговор, ключевым вопросом которого был вопрос о том, что я является ключевым качеством для воина, -- мужество, но как его приобрести? Настоящее мужество -- не разумность без добродеятели.

Рассуждение -- это путь от мнений к знаниям, возможно, проходимый путём индукции. <<Познай самого себя ($\gamma \nu \varpi \vartheta \iota \sigma \varepsilon \delta \upsilon \tau o \nu$)>>, -- говорил Сократ. Для него тело -- это всего лишь инструмент, правильно осуществляющий желания Души. Сократ был уверен, что душа становится лучше, когда она приобретает знания и добродеятель (некие качества, которые помогают правильно относиться ко всему). Философ был уверен, что люди плохо устраивают жизнь, потому что не понимают, как правильно отличать добро от того, чем добро не является.

Сократ был уверен, что мир устроен Божеством. Глаз человека устроен, чтобы видеть, но должно быть некоторое разумное начало, которое его устроило; то же самое и с разумом: разум не может появиться и чего-то более низшего, чем он сам, -- разум дан человеку от Бога.

 Сократ закончил свою жизнь, будучи казнённым в Афинах. Демократы в Греции инициировали процесс против Сократа, который принял некие решения, которые боролись с интересами действующей власти. Этот процесс был задокументирован Платоном в его первом произведении <<Апология Сократа>>, на нём Сократ сам защищал себя на ней, поэтому мы много можем знать о его характере и верованиях. На вопрос: <<Кто самый мудрый среди эллинов?>> -- Дельфийский Оракул, назвал имя Сократа, который ответил, что он знает только то, что ничего не знает; самое сложное занятие, по мнению Сократа, -- это управлять людьми. Сократ беседовал с этими людьми (не можешь сделать своих детей достойными людьми -- какой ты управленец), поэтами (красиво говорят, но понимают ли, о чём?) и ремесленниками (разбираться в одном аспекте чего-то -- не значит считать себя умным) и понял, что они не видят меру своего незнания.
 
Критика Сократа софистами. Некий софист считал, что мужество и разум -- это инструменты для осуществления своих желаний. Сократ же ему ответил, что такой человек -- это дырявая бочка; Но если сосуд полный -- это каменная, устоявшаяся жизнь, однако больной часоткой подходит эти определения [без здорового тела невозможно удовлетворение своих желаний -- то же и с душой]. Нужно работать над душой. А то, что истины нет, -- это истина? 

\section*{4. Заканчиваем по Сократу. Философия Платона}

Сократ считает самым важным то благо, которое свяано с душой, например, знания и истина. Ими можно делиться и они умножаются, будучи равнодоступными всем. В этих благах Сократ видит путь к человеческому счстью. Наслаждение же философ считает неким дополнением к счастью. Идеал жизни -- добродеятельная разумная жизнь. Ровно так же, как человек в знании схватывает что-то универсальное, так и единый божественный разум устроил и природу -- в ней всё целесообразно.


Если подводить итог влияния Сократа для последующей философии, то он ввёл в эту науку индуктивный метод и последовательное определение, полагая, что нельзя ничего высказать о предмете, не давая его понятия, сущности предмета.
Для софистов у понятий нет объективного содержания, но задача человека перейти о мнений к знаниям. Также Сократ выдвинул на первый план проблему самопознания человека -- познания природы своей души, её естественных потребностей, целей. Для Сократа человек -- некая тайна. Поэтому и философия Сократа стала поворотным пунктом в истории античной философии -- она обратилась от познания природы обратилась в изучению человека как разумного существа.

После смерти Сократа появились т.н. сократические школы, для которых Сократ был образцом философа, мудреца. Представители этих школ на первый план выдвигали нравственные вопросы, желание быть добродеятельным. Но на этом и сходство и заканчивалось. Рассмотрим 2 такие школы.

\subsection*{1. Сократические школы}
\subsubsection*{1. Школа Киников}
Эта школа просуществовала около 600 лет. Название она получила от гимнасия Киносарг, где проходили собрания членов этой школы. Основателем этой школы является Антисфен. Он слушал беседы Сократа.

Киники особенно восприняли умение Сократа обходиться малым в жизни: <<Иные люди живут, чтобы есть, а ем, чтобы жить, >> -- говорил Сократ. Мудрец умеет извлекать пользу из любых обстоятельств. Киники говорили: <<Чем меньше человеку нужно, тем ближе он к Богам>>, и отсюда избрали свой особенный образ жизни, у них не было своего места жительства, хорошего одеяния. Но в этом они видели свободу от привязанностей. Киники создали даже целый жанр сатирической литературы, где высмеивали человеческие пороки.

Одним из известных киников был Диоген, который жил в каменной бочке. Как-то Диоген пришёл на пир, где был другой ученик Сократа, Платон, который заметил: <<Сквозь дыры твоего плаща, Диоген, светится тщеславие>>. Стремление эпатировать публику своим внешним видом и поступками было очень важным. 

Впоследствие на латыни это слово записывалось, как cynici (первая буква произносилась, как k). Вот эти их произведения, где они осмеивали все пороки человека, там не было тепла, сочувствия другим людям, это был смех такого превосходство. Настоящее совершенство же готово помочь всем подняться до своего уровня.

Киники не были на центральной дороге философии. 

\subsubsection*{2. Школа Киренаиков}
Параллельно со школой киников развивалось и движение киренаиков. Название происходит от города Кирены, расположенного в северной Африке. Основателем этой школы был Аристипп. 

Касательно нравственности киренаики считали, что цель жизни -- это счастье. Они считали, что человеку недоступно знания, -- реального знания нет, человек замкнут в знаниях своих ощущений. Значит, не надо искать объективную меру, следовательно, надо искать больше приятных ощущений и избегать неприятных; такая политика получила название гедонизма. Именно от киренаиков пошло <<Только настоящее наше>>, аналог латинского carpe diem. Они понимали это в смысле того, что благо приносит общее переживаемое чувство наслаждения. Нравственный нормы киренаики подобно софистам объявляли предрассудками.

Среди предствителей этой школы выделим Гегесия, прововедника самоубийств. Логика такая: если человек видит счастье только в получении наслаждения, то находит, что они потом приносят старадания по тем или иным причинам. Старадаешь, чтобы заполучить наслаждения, потом старадаешь в поисках новых. Если так уж тяжело быть счастливым -- цель неосуществима, то, возможно, проще совершить самоубийство.

Киренаики опробовали т.н. путь безоглядного гедонизма, неслучайно эта школа не имела продолжения.

\subsection*{2. Платон -- главный ученик Сократа}

Имя Платон -- это литературный псевдоним (буквально <<широкий, широкоплечий>>). Изначальное имя при рождении -- Аристокл. Годы жизни 427 -- 347 до н.э. Платон происходил из очень знатного рода: был потомком последнего Афинского царя, учень уважаемого, по одной линии, а по другой -- был потомком Солона, Афинского законодателя. В юности интересовался политикой, ибо его родственники, Критий и Хармит, входили в состав правящих тогда в Афинах 30 тиранов; пробовал писать стихи.

Всё изменила встреча с Сократом, и Платон решил посвятить жизнь философии. После Сократа возник даже такой жанр в греческой литературе, как диалог, когда повествование ведётся как беседа нескольких лиц. Во всех, кроме последнего, диалогах Платона главным действующим лицом является Сократ. Но отсюда сразу вопрос: где в этих диалогах заканчивается Сократ и начинаются собственные мысли Платона. 

Если говорить об учении Платона, то центральное место занимает учение об идеях ($\varepsilon \iota \delta o \varsigma$  -- <<вид>>). С точки зрения Платона, нужно различать мир становления (Гераклитовский мир, где всё течёт) и мир бытия. Мир становления -- это мир чуственно воспринимаемых вещей, а мир бытия -- мир идеи. Идеи вечные, они всегда одни и те же, они совершенны. 

Пример: корабль. Он каждый ден незаметно меняется, может и сгореть, и утонуть. Однако идея корабля: его конструкция, назначение и прочее никуда не денется.    
У Платона к миру бытия также относятся и числа.

К первому миру относятся вещи, воспринимаемые чувствами, а ко второму -- числа и идеи, воспринимаемые умом. Ум, по Платону, -- это тоже своеобразное зрение: когда человек понимает, он тоже своеобразно видит. 

Идея даёт предствление о целом, оно даёт знание, как соединять части.

В ранних диалогах Платона Сократ задаёт вопрос тем или иным собеседникам, пытаясь найти идею чего-либо. Платон впоследствие считает, что в мире бытия есть некое трансцендентное первоначало (единое, благо), которое выше его понимания. Таким образом, Платон выстраивает некую линию устройства нашего знания:

\begin{itemize}
\item Мир умопостягаемый (ЗНАНИЯ):
\begin{itemize}
\item идеи, постигаемые умом;
\item дискуссивная способность рассудка, позволяющая выводить одного из другого (пример: геометрия) -- место науки в наше время;
\end{itemize}
\item Мир чувственновоспринимаемый (МНЕНИЯ):
\begin{itemize}
\item вещи (природа) -- чувства;
\item видимость (область образов, фантазий, иллюзий) -- воображения;
\end{itemize}
\end{itemize}

Платон объяснил, что ложь возникает, когда иллюзии выдаются за реальность.

Цель человека: двигаться от мнений к истине.

Наряду с единым, с миром бытия, существует ещё и начало материи, которая даёт вещи, которые существуют временно, что изменчиво. У Платона есть образ фигурок из металла, для конструирования которых нужна идея, а материя металла она есть и так.

\subsection*{1. Антропология Платона}
Что есть человек по Платону?

Человек состоит из двух частей: душа (относится к миру идей, её нельзя увидеть, как не кромсай человека), тело (постепенно меняется, относится к чувственно воспринимаемому миру).

А если в корабле изменить все детали, будет ли этот тот же самый корабль? С точки зрения материи, нет. То же и с человеком. Человека можно узнать по душе, она может и приобретать знания, познавать мир бытия, но идея будет та же.

Душа имеет дело с миром идей: <<Подобное познаётся подобным>>, -- как говорим Эмпидокл. Она имеет три уровня (три способности):
\begin{itemize}
\item разумная (ум);
\item яростная (эмоции); 
\item вожделеющая (желания, ближе всего  к телу);
\end{itemize}
(см. диалог "государство")

Человек, по Платону, находится в праивльном состянии, когда правит разумная часть души, яростная -- союзница разумной, а не вожделеющей, а последняя -- управляемая.
Если главная вожделеющая, то человек в рабском состоянии своих жеданий.
Данную картинку можно представить в виде колесницы с двумя конями, чёрным и белым.

Душа -- бессмертна, при физической смерти просто разрывается связь между телом и душой. Душа нечто неделимое, а материя распадается.

\subsection*{2. Диалог <<Государство>>}
По аналогии со способностями души, существует три сословия:
\begin{itemize}
\item правители-мудрецы (философы -- люди, любящие мудрость) [добродеятель -- мудрость];
\item воины-стражи [добродеятель -- мужество];
\item все остальные [добродеятель -- умеренность];
\end{itemize} 

у первых двух сословий не должно быть ничего личного, они должны служить общему, служить во благо государства, их нужно специально воспитывать, готовить к службе;
у него ключевая дисциплина -- математика, самая умопостигаемая наука. 
Когда все не меняют свои системы ценностей, устанавливается справедливость.


Платон сравнивает государство по системе ценностей.

\begin{itemize}
\item Самое совершенное по Платону -- аристократия (<<власть лучших>> == власть мудрых, кого считать лучшим -- большой вопрос) [главное -- мудрость, истина].
\item Тимократия (<<власть чести>>) [главное -- доблесть, мужество]. Уже сами не ищут мудрость. Образец -- Спарта.
\item Олигархия. [главное -- богатство]. Богатство даёт независимоть, становится не важно, как ты приобрёл богатство, как ты им распоряжаешься, главное -- нести прибыль. Люди, которые ничего не делают, -- трутни по Платону, они рождают бурю в государстве.
\item Демократия (каждый живёт как хочет -- вожделеющая чать души активная, человек становится неустойчивым) [главное -- свобода]. Возникают явления, что учителя боятся учеников -- притесняется демократия, всё принудительное -- возмутимо. Демократия создаёт почву для Тирании.
\item Тирания (полностью доминирются вожделеющей частью души, а яростное начало -- его союзник. 
Согласно Платону, демократический строй формирует собой предпосылки для ещё более деградирующего общественного устройства -- тирании. Это связано с тем, что в демократическом обществе формируется демократический человек, который балансирует между своими желаниями --  всё, что является желанием, достойно осуществления. Прослойка людей, порабощённых своими желаниями, готовыми на всё, формирует базис тирании.

Сравнивая человеческую душу с Аркополем, Платон пишет о процессе формирования тиранического человека; если человек попадает в общество похожих на него, особенно молодых порочных людей, внутри человека начинвется борьба. 
Душевное богатство умений и знаний подобно Акрополю, а такая компания -- штурм этого Акрополя. Если этот Акрополь оказывает недостаточно защищённым, то вожделения вносят в душу человека <<ложные мнения, хвастливые речи>>. Внутри человека происходит изменеия взглядов на всё. Отрицательное, например, наглость и безрассудство начинает восприниматься как геройство. 

\end{itemize}

Ещё одна тема, затронутая в диалоге, -- тема значения искусства для общественной жизни, где его социальная философия переплетается с эстетикой. 

Он исходит от распространённой античной концепции: искусство возникает от подражания. Платона задаётся вопросом, чему подражают поэты и драматурги, когда изображают человека. Размышляя над этой проблемой, он приходит к выводу, что поэты, не будучи сами приверженцами размной стороны, изображают в человеке его неразумную сторону, то есть его яростное и вожделеющее начала -- такие герои имеют успех у толпы. Философ полагает, что когда поэт обращается к такому яростному и переменчивому нраву, то это нравится зрителям, хорошо поддаётся воспроизвдению; но тем самым поэты питают и укрепляют эту часть души человека, никак не питая разумную часть, и этим самым дают зрителям образцы для подражания. Платон относится к [поэтическому] воображению очень осторожно и даже отрицательно, что характерно для античной философии, потому чтто воображаемые предметы не обладают полноценным бытием: воображение коснструирует иллюзорные, обманчивые образы.  То же и применимо по отношению к музыке.

\section*{5. Аристотель -- ученик Платона}

Аристотель (384 -- 322 гг. до н.э.) был родом из Македонии, провинциальной части северной Греции, происходил из семьи придворного врача.
О нём Платон писал: <<Если для некоторых учеников нужны шпоры, чтобы заставлять их учиться, то для Аристотеля нужна узда>>. Аристотель стал образцом учёного. 

Первые произведения Аристотеля (диалоги, подобные Платону, и какие-то произведения) не дошли до нас, зато дошли конспекты лекций, которые он читал в своей школе. 

Аристотель около 20 лет учился в академии Платона, занятия там были неким образом жизни: были и учёба, и исследования. После учёбы Аристотель по приглашению Македонского царя Филиппа едет быть учителем для его сына, Александра. Однако учёба была не долгой -- Филиппа убивают. После этого философ возвращается в Афины и созадёт там собственную философскую школу <<Ликей>> в 335 г. до н.э. Ещё другое название этой  школы -- перипатетики -- <<прогулка с беседой>>. Его интересовали различные разделы науки от физики до социальной философии. Потом после смерти Александра в Афинах вспыхивает восстание, Аристотетль вынужден был бежать. 

\subsection*{1. Классификация знания по Аристотелю}

Согласно Аристотелю, опыт отличается от знания тем, что знающий отвечат на вопрос: <<что и почему?>> -- то есть знает причины тех или иных событий	и способен эти знания передать другому.

Виды знаний [по важности]:
\begin{itemize}
\item теоретические (<<обозревать>>) знания (науки): первая философия (изучает сущее в целом, первопричины всего), вторая философия -- физика.
\item практические (<<действовать>>) науки: этика, политика, экономика (законы ведения домашнего хозяйства)
\item пойетические (<<творчество>>)
\item Логика по Аристотелю -- законы мышления и доказательства
\end{itemize}

\begin{table}[]
\begin{tabular}{lll}
\textbf{Виды знания (науки)} & \textbf{Предмет}  & \textbf{Цель}                                                           \\
\textit{Теоретические}     & сущее (независимо от нас)   & ради знания                                                    \\
\textit{Практические}       & деятельность {[}человека{]} & \begin{tabular}[c]{@{}l@{}}правила для\\ действий\end{tabular} \\
\textit{Пойетические {[}творческие{]}} &
  \begin{tabular}[c]{@{}l@{}}создание того, чего нет\\ в природе\end{tabular} &
  \begin{tabular}[c]{@{}l@{}}средство для \\ деятельности\end{tabular}
\end{tabular}
\end{table}


\subsection*{2. Онтология}

Аристотель, анализируя понятие сущего -- что есть, приходит к выводу, что мы используем это понятие в разных значениях. Он вводит в обиход понятие категории -- базиса, с помощью которого мы описываем и классифицируем сущее.	

Аристотель относит к сущему:
\begin{itemize}
\item[1.] сущность [отвечает на вопрос <<что?>>], например, человека,
\item[2.] качество [отвечает на вопрос <<какой?>>], например, молодой человек,
\item[3.] количество [отвечает на вопрос <<сколько?>> и прочие], например, рост этого человека,
\item[4.] отношение [отвечает на вопрос о взаимосвзяи], например, брат, ученик и проч.,
\item[5.] место [отвечает на вопрос <<где?>>], например, в Греции,
\item[6.] время [отвечает на вопрос <<когда?>>],
\item[7.] положение в пространстве,
\item[8.] обладание/лишённость, например, вооружённость,
\item[9.] действие,
\item[10.] претерпевание [пассивный или активный залог действия].
\end{itemize}


Первая категория -- самая главное, остальное -- её характеристики.

Сущность есть первая и вторая: вторая сущность -- человек, а первая -- конкретное существо. Первая сущность -- подлежащее в предложениях, не являясь характеристикой чего-то другого. Вторая сущность -- идея по Аристотелю. Если по Платону идея -- это нечто полоноценное, существующее независимо от чего-то другого, а вещи -- тени идей. А у Аристотеля вторые сущности не существуют без первых. 

А как описать первую сущность без второй? Аристотель вводит определение: первая сущность == материя (некий субстрат) + форма(идея, вторая сущность); пример медного шара: существует материальный шар, только один, отдельно от математической абстрации шара он не существует.
Чтобы появился такой шар, должна быть:
\begin{itemize}
\item материальная причина (из чего),
\item формальная причина (что?),
\item движущая (от чего? что приводит к этому?), 
\item целевая (ради чего?) [Природа не похожа на трагедию].
\end{itemize}

Ещё одно определение материи по Аристотелю -- $\delta \upsilon \nu \alpha \mu \iota \varsigma$ (от него -- <<динамика>>, сила, возможность [чем-то быть]). Чтобы что-то реализовалость (первую сущность), нужна $	\varepsilon \nu \varepsilon \rho \gamma \varepsilon \iota \alpha$ (<<действительность>>). Любая вещь действительно существует только тогда, когда она действительно функционирует в соответствии со своей природой.

Аристотель ставит вопрос: есть ли такая действительность, где нет материи, одна форма? По Аристотелю существует мировой божественный разум, который постоянно мыслит, занимаясь чистой разумительной (высшей по Аристотелю) деятетльность. Божественный разум -- форма форм. Без божественного разума по Аристотелю не могла бы существовать природа. Действительность первее возможности.

\subsection*{3. Этика}

Когда речь идёт о человеке, его совершенство проявляется в его деятельности. Человек, по Аристотелю, $\zeta \omega o \nu$ $\lambda o \iota \gamma o \nu$ $\kappa  \alpha \iota$ $\pi o \lambda \iota \tau \iota \kappa o \iota $ : <<Живое существо разумное и социальное>>. 

Благо души -- это добродеятели.

Этика Аристотеля продолжает традиции Сократаи и Платона, считая, что цель деятельности -- это благо. Виды благ по Аристотелю:
\begin{itemize}
\item внешние (мы ими можем обладать): богатство, власть, почёт;
\item телесные: здоровье и чувственная красота;
\item душевные: добродеятели Сократа -- самое главное по Аристотелю:
\begin{itemize}
\item разумная часть: мудрость, хорошая память, благоразумие и т.д. [путём обучения]
\item неразумная часть (не менее важная, чем разумная: она может и препятствовать творению добра, нужно разобраться в этих понятиях): наш нрав -- этос, наши привычки, терпение, мужество [путём упражнения, вырабатывающего устои душу].
\end{itemize}
\end{itemize}
Нравственные добродеятели -- некая середина между двумя крайностями -- пороками. Правильно -- это когда мы соответствуем некой мере. Например, мужество, одна крайность -- трусость, другая -- безрассудная смелость, что даёт неразумное поведение. Нельзя по книгам стать мужественным, мы изучаем этику не для того, чтобы узнать о добродеятелях, а чтобы стать добродеятельным.


\section*{6. Продолжение об Аристотеле, эпоха эллинизма}
\subsection*{1. Социальная философия}
Аристотель говорил, что человек -- социальное существо, существующее в той или иной форме общения.
Аристотель выделяет три уровня общения:
\begin{itemize}
\item семья;
\item община;
\item государство.
\end{itemize}
По Аристотелю, полноценное существование человека связано именно с государством. Вырабаютывают определённые типы отношений между людьми: если между взрослыми отношение более-менее равные, то к детям было царственное отношение, как и рабам. 

Задачу государства Аристотель видит в том, чтобы помочь человеку стать добродеятельным.
Что касается видов государств, они могут различаться по типу правления, как, например, монархия и тирания, аристократия и олигархия, республика и демократия (охлократия) -- правит либо один человек, либо какое-то меньшинство, либо большинство.

Для Аристотеля не важно, какая часть общества правит, но если цель государства заключается в том, чтобы принести человек благо, то это позитивный ряд (первые термины). Второй ряд негативный, потому что власть начинает думать о собственном благе. Аристотель считал, что мало добродеятельных людей, поэтому обществу более подходит аристократия и монархия.

Аристотель проводил аналогию с искусством: если человек безнравственный, то это его ответственность, что он не может над собой работать, чтобы исправить негатиные устои души.

\subsection*{2. Экономика}

Аристотель отделял 2 науки: экономику и хрематистику. Если экономика -- наука об устройстве домашнего очага, об управлении хозяйством, то хрематистика -- это про деньги. Цель хозяйственной деятельности -- это удовлетворение естественных потребностей человека. Но там присутствуют деньги как средство, и ведение хозяйства может превратиться в получение прибыли, что, как считал Аристотель, неправильно, потому что когда экономика имеет меру, равную потребностям человека, то существует мера, то в случае денег разумной меры нет -- потребности человека становятся средством для увеличения богаство. Средство и цель меняются местами.


\subsection*{3. Эпоха эллинизма}
После смерти Аристотеля наступает эпоха эллинизма, из-за завоеваний Александра Македонского греческий язык распространяется по всему средиземноморью и Средней Азии. В этот период философия воспринимается как путеводитель к счастью. Разные философские школы давали ответы на этот вопрос.

Самыми известными были следующие философские школы:

\subsubsection*{1.Эпикурейцы}
Основателем был Эпикур(341 -- 271 гг. до н.э.). Он основал примерно в 306 году до н.э. школу под названием <<сад Эпикура>>. Основная идея: природа уже знает, что такое счастье, -- это наслаждение. Но в отличие от киренаиков, эпикурейцы стараются именно избегать страдания. 

Эпикур отрицательно относился к образованию и любому воспитанию, рассматривает добродеятели только как средство к достижению удовольствий, давая следующую их классификацию:
природные [необходимые и не необходимые] и неприродные(искусственные), например, стремление к славе. <<Эпикур учил об устойчивом беспердметном самонаслаждающемся человеческом организме, взятом как целое,>> -- говорили об Эпикуре. Но такая философия была не для всех. 

Эпикур учил, что всё состит из атомов, душа тоже, смерть -- это простой распад, переживать об этом не стоит. У него отсутствует стимул созидательной деятельности. 

В области гносеологии он был сенсуалистом, то есть исходил из чуственных ощущений и опыта. Исходя из этой предпосылки, Эпикур не отрицал существование богов: нет дыма без огня, но он считал, что Боги -- тоже эпикурейцы, наш мир их вообще не интересует.

Учение эпикурейцев было не очень долговечным, как и движение киренаиков, потому что у человека отсутствует цель в жизни, кроме как убежать от страданий.

\subsubsection*{2. Скептики	}
Их основателем считается Пиррон (360 -- 270 гг. до н.э.). Пиррон сформулировал 3 вопроса:
\begin{itemize}
\item В каком виде существуют вещи? не существуют, неразличимы
\item Как мы должны относиться к ним?  не можем доверять ни разуму, ни чуствам
\item Как мы должны жить, исходя из вышесказанного? не высказываться о вещах, воздерживаться от суждений. 
\end{itemize}


Для скептиков всё $\alpha \delta \iota \alpha \varphi o \rho o \eta$ -- безразлично. От чего старадают люди? Они связывают себя с какими-то истинами, вынуждены перестраивать свою жизни под  них, а с точки зрения Пиррона, ничего толком не известно, тогда никакой гарантии правильности их поведения нет. Скептики (от слова <<рассматривать, исследовать>>) делили всех на две большие группы: одни сами скептики, а вторые -- догматики (всякий, кто имеет убеждения). Если не иметь убеждения, то можно достигнуть бесстастия, апатии ($\alpha \pi \alpha \vartheta \epsilon \iota \alpha$). Как жить -- плыть по течению, ни от чего ни разочаровываться. 

А сама позиция скептицизма -- это убеждение, или нет? Это говорит о непоследовательности скептицизма. О скептиках мы знаем благодаря труду античного автора Секст Эмпирик. Его труды начинаются со слова <<Против...>>, где он всех критиковал. Наследие скептиков -- это критика. Основные методы опровержения:
\begin{itemize}
\item  все живые существа различны, имеют неодинаковые представления об одинаковых вещах;
\item люди так же различны между собой, наши утверждения действительны только относительно нас самих;
\item наши суждения зависят от нашего состояния в каждый момент жизни;
\item наши суждаения зависят от обстоятельств, положения объектов в мирe;
\item всё, с помощью чего мы доказываем, надо доказать.
\end{itemize}
-- всё существует относительно чего-то другого (философский релятивизм).

\subsubsection*{3. Стоики}
Основателем этой школы был Зенон из Китиона (336 -- 264 гг. до н.э). Он был торговцем. После крушения своего корабля, оставшись в живых, он прибыл в Афины и зашёл в книжную лавку. Увидев там книгу воспоминаний о Сократе Ксенофонта, он спросил, где можно найти фиолософа.

Стоическая ($\sigma\omega\alpha$ -- <<портик>>) философия имеет на себе отпечаток движения киников. Движение названо по месту собраниия.
Три периода стоицизма:
\begin{itemize}
\item Древняя стоя: 4 -- 3 вв. до н. э. Главный теоретик -- Хрисипп; стоики разработали пропозициональную логику;
\item  Средняя стоя: 2 -- 1 вв. до н.э. (Панетий). Стоицизм сближается с платонизмом, начинается образовательная деятельность;
\item Римский стоицизм (Эпиктет, Сенека Младший, император Марк Аврелий). Становится публицистической философией.
\end{itemize}

Стоики представляли философию как куриное яйцо: скорплупа -- логика, белок -- физика, желток -- этика (то, на что работают другие части). Стоики были сенсуалистами, душа человека при рождении -- чистая доска, а впечталения оставляют отпечатки на душе, формируется разум.

В области физики стоики были пантеистами (всё есть Бог). Веселенная, по их мнению, -- это существо, в котором всё упорядоченно взаимодействует, с разумом ($\pi \nu \varepsilon \upsilon \mu \alpha$ -- <<дух>>). С точки зрения стоиков, он пронизывает всё мироздание. В человеке пневма -- гегемон, руководящий орган для человека. 

Поскольку весь мир управляется божеством, он человека в этом мире ничего не зависит, но не надо переживать, потому что божественный разум устроит всё как можно лучше. А что же зависит от человека? быть добродеятельным (разумным) или порочным. В области добродеятелей стоики разработали учение об аффектах -- что мешает быть разумным: страсть наслаждения и страдания -> вожделение, порабощающие разум, и страх.

Если говорить об этике, то Стоики исходили из того, что единственное подлинное благо -- это добродеятель, или то, что полезно для высших ценностей жизни. Единственное настоящее благо -- это добродеятель, а настоящее зло -- нравственный порок. Добродеятель стоики понимали как совершенную степень правильно действующего разума, то есть того, кто действует наиболее разумно.

Классификация добродеятелей и соответстующих им пороков:
\begin{itemize}
\item умеренность $\leftrightarrow$ страстное желание, которое приводит к необузданности; человек старадает от своей неупорядоченности, нарушения собственного же приниципа меры;
\item мужество $\leftrightarrow$ трусость (аффект страха);
\item рассудительность $\leftrightarrow$ невежество, неразумие;
\item справедливость -- добродеятель, выводящая человека на общественный уровень, при условии, что развиты три предыдущие добродеятели.

Справедливость -- это когда каждому воздаётся в соответствии с его заслугами и состоянием. Эта добродеятель выводит человека за пределы эгоистического стремления к самосохранению. Эгоизм сам по себе -- не совершество, разумная природа человека стремится к общественному благу.

\end{itemize}

Эти добродеяти полагают в качестве важного основания правильные представления о добре и зле: человек вдаётся в пороки, поскольку не имеет правильных представлений о добре и зле, человек должен культивировать в себе правильные представления о добре и зле, воспитывать себя.


У стоиков формируется этика гражданского долга: у мудреца личные и общественные интересы совпадают. В частности, в Римской империи гражданам импонировала этика, которая была нацелена на служение обществу. 

В эпоху римского стоицизма мы начинаем встречать примеры практической этики: человек из опыта своей жизни даёт некоторые советы и уроки. В качестве примера можно привести Эпиктета. Он отмечает, что <<привычка всякая усиливается от упражнений и укрепляется>>, точно так же в нравственной сфере важны привычки: <<когда ты сердишься,>> -- отмечает Эпиктет, --  <<знай, что ты делаешь не только то зло, но и усиливаешь привычку к гневу [в следующий раз будет сложнее]>>. Так и с дургими вредными привычки: кроме телесных болезней, есть и болезни души, с которыми нужно бороться.

 Эпиктет предлагает наблюдать за собой, за проявлениями своих привычек. Он предлагает бороться с гневом в самом зародыше, обдумывать все мысли, отсеивать плохие, быть в обществе подобных благодеятельных людей. Также он предлагает читать полезные книги мудрых людей, живших до этого -- чтение должно быть полезным.
 
Другой, приводимый римским стоиком пример, -- человек, который играет на музыкальном иструменте, страдающем от беспокойства, что всё не находится в его власти, например, похвала от слушателей. Он не может на это повлиять, боится, что не получится, и волнуется. По Эпиктету, зря волнуется, ибо самое главное делать отлично то, что зависит от тебя, ибо не знаешь <<дешивизну и гнивость похвал людских>>. Не нужно быть рабом такой цели.

\section*{7. Средневековая философия}

Средневековье не имеет каких-то опеределённых границ, они проходят по мировоззрению: античное мировоззрение можно назвать космоцентризм, а средневековое -- теоцентризм (от греч. $\Theta \varepsilon o \varsigma $ -- Бог). 


\subsubsection*{1. Онтология}

В онтологии есть такое важное понятие абсолюта -- того, что не зависит ни от чего другого. Если сравнивать античную и средневековую философию в целом, то главным отличием будет то, как понимается и трактуется абсолют.

Фактически, для античного мировоззрения такиим началом является космос. Античные боги как бы встроены в космос, не имеют власти над ним, для античных людей характерен некий фатализм. Античные трагедии обыгрывают тему рока, его неизбежности. Политеизм, или пантeизм -- [божественный разум -- душа мира по стоикам], дуализм [Платон, Аристотель, в своих глубинах мир от Бога не зависит].

Для средневековых людей более характерен монотеизм, для них Бог -- творец мира, креационист. Материальный мир целиком и полностью создан Богом, полностью от него зависит -- тогда материальный мир хорошо подчиняется жизненным законам, следовательно, изучаем. Мир возникает не в целях необходимости, а по желанию и прихоти Бога. Бог свободен (нет закона, который выше его), в этом смысле Бог является личностью -- разумным существом, обладающим правом и свободой выбора.

В философском смысле существует некая онтологическая грань между Богом и творением. Бог вечен, а мир имеет начало. 
Ещё одной важность чертой был ревеляционизм(откровение): поскольку Бог не зависит от мира, то естественнным путём мы не можем познать его, как мы познаём мир, ибо между ним нет никакой связи. Мы естественным путём можем прийти к выводу, что Бог существует, открывая некие свои черты личности, но не себя полностью. Для человека это важнее, чем само изучение мира. 
 
 
\subsubsection*{2. Философская антропология}
В иерархии существ, которые населяют этот мир, человек занимает промежуточное место. Цицерон о природе Богов в своей работе пишет: <<Ведь если Богов нет, то что может быть во Вселенной лучше человека, ибо только в нём нет разум, превосходнее которого ничего не может быть. Но было бы безумным высокомерием со стороны человека считать, что лучше него во всём мире ничего и никого нет>>. 

В средневековом мировозрении, когда речь идёт о творении мира, Бог сначала творит разумные существа, называемые ангелами, разум которых выше, чем у человека, --  в этом смысле они ближе к Богу, -- но недоступны для чувственных восприятий. Затем Бог творит видимый мир, где человек -- ключевая фигура, к него есть тело и душа: тело принадлежит к видмиому миру, душа -- к невидимому, и человек в отличие от растений и животных, содержит в себе образ Бога [в разумности, в самовластии (в отличие от других живых существ, человек имеет власть выбирать, слушаться Бога или нет)б в бессмертии души], который ставит его выше над всеми другими живыми существами. С точки зрения средневекового человека, Бог создал его, чтобы человек стал существом, подобным Богу [ряд авторов различали понятие образа Бога -- заложенного изначально-- и подобия Бога -- цели]. 

В Платонизме связь души и тела -- зло, нужно скорее порвать эту связь (конечно, не путём самоубийства), но для средневековья это не так: тело тоже создано Богом, поэтому был популярен обычай воскрешения тел. Тогда перед Богом все, получается, равны, устанавливается некое равенство перед Богом. Отсюда и учение о падении души: первые люди отпали от Бога, исказилось их внутреннее состояние, их души стали, как растроенные музыкальные инструменты. 

После падения первых людей жизнь человека стала делиться на 2 стадии: телесная(временная) жизнь (от рождения до смерти тела), смерть до воскресения тел, вечная жизнь. В этом смысле временная жизнь дана человеку, чтобы подготовиться ко встрече с Богом,  разделяя позицию античной этики о добродеятелях. Иоанн Златоуст говорил, что Бог дал человеку прекрасное тело, а его задача -- сделать прекрасной свою душу.

Важным является понятие Суда. Бог, зная жизнь о каждом человеке, представляет его к ответу за его слова и поступки, постепенно разделяя людей для вечной жизни. Плавно перешли к понятию эксхатологии ("последний") -- философии истории [в отличие от античного мира, который представлял себе прогресс как круг, в Средние века это уже направленный вектор]

Средневековье рассматривает аффект как злоупотребление искажёнными силами своей природы.

\subsubsection*{3. Периодизация средневековой философии}

\begin{itemize}
\item 2 -- 3 вв. н.э. -- период апологетов (апология -- <<защита>>). Выступали в защиту своих собственных убеждений от античного общества и философии. Известные представители: мученик Иустин Философ, Тертулиан, Климент Александрийский
\item 3 -- 8 вв. -- патристическая философия: греческая [Василий Неокессарийский (4 в), Григорий Назианзин и Иоанн Златоуст] и латинская []
\item 9 -- 10 -- 13 вв. -- Арабская средневековая Философия, в западной Европе появляется схоластика, Исихазм -- в Византии.
\end{itemize}

\subsubsection*{4. Апологеты}

Иустин Философ -- один из самых крупных представителей движения апологетов, был платонистом, стал христианом. Он пишет две апологии, обращаясь к римским императором, разговаривая с ними на языке философии, он пишет: <<Каждый человек носит в себе семена божественного Логоса, и философия ниспослана людям от Бога>>. С его точки зрения, истина философии распалась на множество школ, много <<голов>>, что неправильно, а те язычники, которые учили добродеятелям, подвергались гонениям. Жизнь христиан в земном мире -- это война, ибо люли изачально разделились; есть падшие ангелы, и они стараются повести часть людей за собой, через мысли и вообржанение принуждают следовать за своими мыслями. В Христианства можно учиться жить вместе с Логосом. 

Апологеты критиковали политеизм, порицая поведение античных Богов, воспеваемых местными поэтами. Климент Александрийский считал, что зло не может породить добро, поэтому философия, которая учит добру, не может быть таковым, но её надо дифференцировать.

\section*{8. Патристика и схоластика}

\subsection*{1. Патристическая философия}
Слово <<патристика>> происходит от греческого $\pi \alpha \tau \varepsilon \rho$ -- <<отец>> и латинских <<pater, patris>>, то есть буквально означает <<философия отцов церкви>>.

Ранее различали два типа богословия:
\begin{itemize}
\item апофатическое (чем Бог не является);
\item катофатическое (что есть Бог).
\end{itemize}

Признаки принадлежности к отцам церкви:
\begin{itemize}
\item[1.] верность богооткровенному учению [с точки зрения средневековой философии божественное откровение не требует никакого улучшения, потому что оно дано совершенным вечным Богом, и задача тех или иных религиозных мыслителей заключается в том, чтобы его сохранить и передавать последующим поколениям, отсюда и пошло понятие <<ереси>> -- возможности выбирать часть откровения, а остальные искажает];
\item[2.] святость жизни [доверие вызывает человек, имеющий высоконравственные качества, например, мученники, которым доверяли, потому что за верность божественному учению готовы переносить страдания и смерть];
\item[3.] Авторитет в церкви и признание трудов.
\end{itemize}  

В рамках этого определения важное значение имеет понятие <<consensus patrum>> -- согласие отцов: верным признаётся только то, что признавалось всеобщим согласием отцов.

\subsubsection*{1. Греческая патристика}
Яркий пример -- Василий Неокесарийский.

Своего расцвета греческая патристика достигла в IV веке. Жизнь и труды Василия Неокесарийского являются показательными для этого периода философии. Он родился в Кесарии, имперском городе на территории современной Турции, был родом из богатой семьи, получил хорошее образование, внутри него боролись два стремления: жажда знаний и стремление к аскетству, которые определили его последующую жизнь.

Он жил ещё в то время, когда римские императоры не до конца придерживались позиций выше, поэтому претерпевал гонения, сталкивался с имперской властью, но он оставил после себя ряд сочинений, которые вошли в наследие патристики и имеют характерную тематику. 

Многие из его произвдений были в жанре бесед, которые были за ним записаны, но были и представители других жанров, как, например, полемика с Евномием, его современником, о богопознании. Евномий считал, что Бог познаётся через понятие: если мы имеем правильное понятие о Боге, то мы познаём сущность Бога -- позиция, близкая к радикальному рационализму. Василий же выступает с критикой: понятия о Боге не раскрывают нам сущность Бога, имена Бога раскрывают нам только его действия. Аналогия с изучением природы: понятия о вещах дают нам только представления об их качествах, но не сущности. Сущность Бога скрыта от познания человека. 

В других своих трудах, например, <<Внимай себе>> затрагивается тема богопознания, которое имеет 2 формы:
\begin{itemize}
\item[*] естественное -- познание Бога через изучение окружающего мира [Мир похож на большую картину, которую человек, внимательно рассматривая и изучая, может сделать выводы о мудрости и могуществе Создателя], из самопознания человека [Из того, что душа не имеет плоти, Бог тоже не имеет тела, как ум не ограничен местом, так и Бог тоже, он не имеет ни цвета, ни запаха,а узнаётся только по действиям]; 
\item[*] путём божественного откровения (сверхестественное), которое имеет 2 формы: писание и предание. Предание -- критерий правильного понимания писания.
\end{itemize}

Во времена поздней Римской империи распространение получил гностицизм, который соединял элементы античной культуры и современной монотеистической религии (эклектически). Гностики учили, что путём к спасению, освобождению от страданий является некое особое знание, которое открыто Богом, и люди все делятся на три категории: обычные, душевные и пневматики (духовные). Гностики пытались убедить христиан вступить в свои ряды, подписывая именами святых свои произведения. Возникла проблема отделения таких текстов от богооткровенных. Был собран церковный конгресс, на котором был отобран ряд текстов, названных каноническими (содержат божественные знания) и апокрифический (ложный).


Ещё одна тема, поднимаемая философом, была затронута в беседе <<Почему Бог не виновник зла?>> Если Бог сотворил мир, то почему он допускает зло? Василий Неокесарийский различает два понимания зла:
\begin{itemize}
\item то, что считается злом, что вызывает старадания и смерть;
но не всякое страдание -- зло, будь то старадание при лечении болезней, например. В этом же сочинении он подчёркивает важность
души для человека: он живёт как бы 2 жизни, временную, которая заканчивается для тела, а для души жизнь предела не имеет;
\item то зло, которое портит душу, -- вечное страдание, настоящее зло
\end{itemize}

Бог иногда допускает душевшые страдания, чтобы не допустить душевных или остановить их.

Другой вопрос: зачем Бог сделал человека свободным, ведь человек может сам совершать неправильные поступки и губить себя. Свобода выбора -- одно из совершенств, данных человеку Богом. Создателю важно чтобы человек добровольно его выбрал, добровольно любил. Когда человек правильно пользуется своими способностями, это его достояние.

Василий Неокесарийский также отмечает, что Бог также пользуется и злом, когда душа человека от Бога, для того, чтобы человека различными способами привести к благу, но Бог не может спасти человека без согласия на это самого человека.

Также в творчестве Василия Неокесарийского затрагивается тема отношения к античной философии, он считает, что нужно относиться избирательно, если это пригодится в будущей жизни, то принимать суждения, например, о добродеятели.

Ещё одним представителем греческой патристики является Иоанн Златоуст, который был экзегет (экзегетика -- толкование божественного откровения). Он обладал даром оратора, был приглашён в Константинополь, где нажил себе врагов, потому что обличал роскошь императорского двора, нравы -- был поборником нравственности. Он умел ярко и просто объяснять многие вещи. Например, на вопрос, почему Бог не может дать человеку всё, Иоанн приводил такое сравнение: что бы хотел выбрать, хорошее тело и плохую одежду или наоборот; душе тоже нужно движение, она работает, исправляя то, что нам хочется.

\subsubsection*{2. Латинская патристика}
Нашествия варваров на западную Римскую империю, приводившие к её  упадку, и она сильно контрастировала с восточной частью.

Один из самых известных представителей -- Августин Аврелий (354 --430 гг.) Отец Августина был язычником, но хотел дать хорошее образование, а мать его была христианкой. Его с самых ранних лет увлекала <<ненасытная жажда бытия и наслаждений>>. 

Августин написал произведение, породившее целый жанр в литературе, описывавшее её жизнь, <<Исповедь>>. С юности Августин был увлечён театром; потом стал чувствовать неудовлетворённость своей жизнью, на него оказало большое влияние произведение Цицерона <<Гортензии>>, сделавшая анализ жизненных трудностей автора. Книга эта полностью изменила <<настроение>> Августина, ему стали неинтересны честолюбивые стремления, кроме удовольствий плоти, есть удовольствия от познания ума, он стал желать бессмертие и мудрость, науки для него перестали быть средством, а стал получать удовольствие от процесса.

Августин ищет блаженной, счастливой, полноценной, совершенного жизни (De Vita Beata). Это счастье должно быть достигнуто на разумных началах и неуязвимо для иных превратностей жизни. Он писал: <<Переход от плотского состояния к духовному всегда начинается с убеждений ума.>> Ум понял, но воля не хотела переходить. В поисках этой мудрости он встречается с манихеями (философское течение, производное от агностицизма, дуалисты). Ему понравилась позиция манихеев, объясняющая налиичие зла в мире борьбой <<света>> и <<тьмы>>. После 10 лет у манихеев, он перешёл к скептикам (<<скептицизм -- спутник индивидулизма>>), но он его удовлетворился этим учением, ибо он похож на тупик, который далее не ведёт.

 Далее ему попадаются труды неоплатоников (последователей Платона), которые исправили его ошибку -- по его же собственному признанию -- восприятия Бога как чего-то телесного, реально существующего: кроме телесного, существует реальность, которая воспринимается одним умом, и свободные науки возводят чувственные образы к сверчувственным. Правило справедливости, например, присутствует везде, но не по частям, а целиком, а бесконечность Бога не зависит от пространственных характеристик. Но опят же не удовлетворился данным учением из-за политеизма и того, что Бог стал во плоти.
 
На него оказало большное влияние личность Амвросия Медиоланского, который показал, что не всё в Писании нужно понимать буквально. Потом Августин принял крещение и стал епископом. Помимо <<Исповеди>> у него было важное произведение <<О граде Божьем>>, написанное в тот период, когда северная Африка завоёвывалась Вандалами. Августин там рассуждает о философии истории: представляя мир, как два сосуществующих города  (civitas dei), град небесный и град земной, между ними происходит борьба.

 В средневековой философии обсуждали вопрос <<о воли и благодати>>; Августин увидел, как важна в жизни человека воля, потому что его воля не хотела следовать уму, только божья благодать помогает следовать уму, своими силами человек не может быть на пути совершенства, от него лишь завсит, принять эту помощь или нет.
 
Также в средневековой философии поднимается вопрос соотношения веры и разума -- античные философы упрекали средневековых за то, что они много что принимали без доказательства, полагаясь только на веру: человек в жизни часто полагается на веру, сам того не замечая, он верит в результат, на вере держится множество решений человека. Закон: сначала вера, потом знание (как процесс обучения). Предмет веры -- бесконечный Бог.

Поздняя патристика отвечала на вопрос, можно ли отображать Бога: можно отображать то, как Бог к нам явился, но не более.


\subsection*{2. Схоластика}
Период схоластики характерен для западной традиции XI века.
Слово <<схоластика>> происходит от слова <<школа>>, <<досуг>>, школьное богословие. Образование начинает развиваться, на центральное место выходит диалектика, искусство убеждения.

Схоластика ставила перед собой задачи систематизировать божественное откровение и убедить в нём неверующих, например, людей. Диалектика была в этом смысле одинм из главных средств. Возникает культура диспутов как проверка реальных знаний студентов в университете.

Один из первых представителей схоластики -- Ансельм Кентерберийский -- выдвигает полное онтологическое доказательство бытия Бога. Оно основывается на мысли, существует ли нечто больше, <<больше чего нельзя помыслить>>, некий максимум. Если это существует, то можно дойти до максимума, а если нет, то существует нечто, содержащее само существование, а это оно и есть. Впоследствие многие пришли к мысли, можно ли доказать что угодно только с помощью мысли; 

Как можно с помощью Бога что-то доказать, что что-то есть Бог?

Спор в схоластической философии поделил её на 2 лагеря:
\begin{itemize}
\item реалисты (считали, что идеи существуют реально -- ближе к Платону, ум открывает общее); 
\item номиналисты (считали, что идеи существут в виде имён -- ближе к Аристотелю; ум конструирует общее из каких-то единичных понятий).
\end{itemize}

самый известный средневековый философ -- Фома Аквинский, а его учение -- томизм.

\subsubsection*{1. Арабская философия}
Арабы, создав огромное государство, через сирийцев познакомились с трудами Аристотеля.

На арабскую философию через культурную составляющую оказали большое влияние аристотелизм и неоплатонизм (многие труды неоплатоников подписывались именем Аристотеля). В арабской философии Аристотеля назвали Философом.

Наиболее известные филосософы:
\begin{itemize}	
\item Аль Араби
\item Ибн Сина (Авиценна) (980 -- 1037 гг.)
\end{itemize}
Оба они в рамках философии Аристотеля обсуждали тему универсалий, об общих понятиях, как они существуют. С точки зрения Авиценны, универсалии существуют в уме Бога, а не существуют в реальности, но материя придаёт универсалиям какие-то черты. Соответственно, если мы рассматриваем какой-то тип вещей, то каждая отдельная вещь имеет свои черты в силу материальности, но каждая из них содержит нечто общее: унивресалию. Универсалии же познаются и человеком, когда он через опыт анализирует свои знания, обобщая их, находит общее в вещах, то есть находит универсалии.

Авиценна находит доказательства бытия Бога, которые имели место потом и в схоластике, в частности, такое доказательство, что мир сложен, и раз он составлен из каких-то частей, то его существование не является необходимым, раз он не существует по необходимости, значит должна быть причина, по которой он существует. И эта причина есть Бог -- он первоначало, далее которого ничего нет.

Вместе с тем, учение Аристотеля находило конфликты с религиозными взглядами арабских философов, в частности, учение Аристотеля о том, что мир вечный, несоответствовало мусульманской традии. Это разногласие между Аристотелем и религией проявило себя в деятельности арабского философа Ибн Рушда (Аверроэса). Он вошёл в историю под именем Комментатора Философа. Ибн Рушд написал одни из самых подробных и обстоятельных комментариев на труды Аристотеля, настолько уважал и почитал Аристотеля, что высказывался таким образом: <<Выше Аристотеля никто и никогда ничего не напишет>>, -- для этого философа он был высшей точкой человеческой мудрости. Для того чтобы согласовать взгляды Аристотеля с существующей религиозной традицией, Ибн Рушд выдвинул теорию двух истин:
\begin{itemize}
\item[1.] истина религиозная, открытая Богом;
\item[2.] истина философская, она же открывается путём мышления и познания.
\end{itemize}
Аверроэс же считал, что обе истины от Бога, но первая истина оперирует более чувственными образами -- она для простых людей, а истина от Бога в философской форме -- для мудрецов, которые способны мыслить. То, что они не согласуются, предлагается принять как факт, ибо они для разных людей. Эти взгляды не были приняты, ибо отрицание религиозной традиции были восприняты как отрицание самой религии.

\newpage

\subsubsection*{2. Обратно к схоластике}

До XII века схоластика находилась под определённым влиянием Платонизма, тогда как после XII века западные философы получили доступ к трудам Аристотеля.

У Схоластики было два источника:
\begin{itemize}
\item с запада -- Кордовский халифат, Испания (греческий -> сирийский -> арабский -> латынь);
\item с востока -- Византия, через Сицилию
\end{itemize}

Аристотель -- переводили в основном его лекции -- сразу же становится популярным, ввиду особенностей системы образования в то время: оно строилось на чтении лекций и комментариях к ним.

В XIII веке они столкнулись с той же проблемой, что и арабы (проблемы с религией), и теория двух истин нашла продолжение в трудах латинских аверроистов: истина, что мир имеет начало, и истина то, что мир вечен.

Наиболее выдающимся для схоластики стал Фома Аквинский, преуспевший в полемике с аверроистами.	Он исходил из того, что две истины не могут друг другу противоречить, то есть, теория двух истин -- констатация того, что мы должны смириться с каким-то противоречием -- в Боге не может быть противоречия. Свою же задачу Фома видел, что преобразовать учение Аристотеля так, чтобы оно согласовывалось с современными реалиями. Такая концепция получила название томизм.

Его концепция исходила из того, что человек имеет дело с двумя видами истин: естественные(философские), которые открываются человеку при естественном познании, и истины богословские (истины-откровения) -- то, что Бог открыл людям, а другим путём эти истины узнать нельзя, потому что они закрыты от понятия того, каков Бог. По мнению Фомы Аквинского, все философские истины необходимо согласовывать с богословскими. 

Фома и другие его последователи утверждают, что <<Errare humanum est>> -- человеку свойственно заблуждаться, аверроисты переоценивают взгляды Аристотеля. Аристотель только человек. Также Аверроисты не обратили внимание на то, что физика Аристотеля, очень близко переплетённая с его философией, могла быть неверна, потому что, например, многие её положения исходят из опыта.

В Византии в XIV веке развитается такое направление, как исихазм, наиболее ярким представителем которого был Григорий Палама. Исихазм учил о том, как человек может опытно познать Бога путём аскетической жизни. Палама вступил в полемику с Варлаамом относительно вопросов познания Бога. Варлаам считал, что Бог познаётся только через творение, а между ними всегда есть граница. Но с точки зрения Варлаама, человек никогда не может непосредственно иметь отношения с Богом, Бог трансцендентен по отношению к человеку. Единственный способ приближения к Богу -- земные знания, познавая законы божественного разума. Григорий Палама же задал вопрос: <<В учении какого философа можно найти законы божественного разума?>> -- характеристика земного знания всегда гипотетична, оно не даёт и не может дать полного представления о Боге. Также  науки (математика, логика, красноречие) -- это всего лишь инструмент, который в зависимости от намерений человека может служить как добру, так и злу, то есть может и не приближать к Богу без любви к самому Богу.

Таким образом, Палама учит о двух видах  мудрости: мудрость земная (по монаху Варлааму) и мудрость небесная, которая меняет самого человека, приближая его к Богу. Она проявляет себя не в делах, а в словах. Бог проявляет себя в энергиях, к которым может приобщаться человек, изменяясь и приобщаясь к Богу. Этот путь повлиял не только на Византию, но и на Древнюю Русь.

\section*{9-10. Эпоха Ренессанса (возрождения) и философия Нового времени}
\subsection*{1. Ренессанс}
Что же возраждалось в эту эпоху (XV -- XVI вв.)? Возрождалась античная культура. 

Если говорить о предпосылках ренессанса, то первой стоит упомянуть греческую миграцию, то с XI века у Византии появились проблемы с востока -- с турками-османами. Немало греков переселялись на территорию Италии, там происходит вспышка популярности греческой культуры на западе.

Второй фактор -- кризис папства (не было единой папской власти).

Ренессанс имеет три периода:
\begin{itemize}
\item[*] гуманистический [первая половина XV века]
Слово <<гуманизм>> имеет два значения:
\begin{itemize}
\item Цицерон ввёл понятие humanitas = человечность + образованность -- человеческое начало;
\item [наш случай]: антропоцентризм -- визитная карточка ренессанса -- человек трактуется как высшее существо в мироздании, имеющее право на всё, решающий, что есть добро, а что зло, -- контраст с античностью, выросший на средневековой почвой человека как ограниченного образа Бога.

Петрарка вспоминает Августина: <<Человек не найдёт себе полноценной жизни нигде, кроме как в Боге, бесконечном и совершенном>>, но заканчивает фразу по-другому: <<Человеческая душа может остановиться только на самой себе>>. Петрарка воспевает свои чувства, своё Я. Средневековье интересует жизнь людей и Бога, общее благо, то для ренессанса характерен индивидуализм, земная жизнь. 
\end{itemize}
\item[**] неоплатонический [первая половина XV века -- начало XVI века]
\item[***] натурфилософский [XVI век]
\end{itemize}

Характерная черта ренессанса -- эклектизм (некритическое связывание разнородных учений, часто несовместимых друг с другом). Представители ренессанса, будучи индивидуалистами, не связывают себя ни с какой традицией или течением, они по своему усмотрению берут и связывают друг с другом то, что им нравится, но не синтез, а эклектика. Они занимаются всем, проявляя себя в искусстве и творчестве.	 Бог -- творец, следовательно, человек тоже может быть творцом, некое подражание к Богу.

Образ мрачного средневековья -- образ, нарисованный ренессансом, потому что оно непонятно из-за другого мировоззрения: дети ренессанса смеются над схоластическими трактатами, философией и искусством, потому что оно им непонятно, как замечает один из авторов. Ренессансу важно то, чтобы искусство рисовало то, что мы видим глазами, а средневековая живопись была более символичной, в ней не было перспективы, всё имело определённый символ.

Неоплатонический период назвается так потому, что в этот период во Флоренции открывается Флорентийская академия, главным представителем которой был Марсилио Фичино, который переводит труды Платона, неоплатоников -- неоплатонизм становится очень популярным. Если хочется найти изображение ренессанса, то поможет картина Рафаэля "Афинская школа", в центре которой изображены Платон и Аристотель. Первый держит <<Тимей>> -- диалог об устройстве космоса, -- что показывает: ренессансу Платон интересен своей философией мира. Аристотель же держит в руках своих <<Этику>> -- как приобрести земное счастье. Это хорошо символизирует приоритеты ренессанса.

Самым ярким мыслителем эпохи неоплатонизма является Николай Кузанский. Он был родом из Германии, из бедной семьи. В это время идут дискуссии о власти папы, в которых он активно участвует, в конечном итоге становясь кардиналом. В его трудах формулируется идея бесконечной Вселенной, центр которой везде, окружность --нигде: он начинает активно использовать понятие бесконечности в естествознании. Вселенная неким образом отображает Бога. Это учение очень сильно повлияло на Джордано Бруно -- натурфилософа, который зачитывался трудами Кузанского и заимствовал его идею бесконечной Вселенной.

В последний период ренессанса становятся популярными философские учения о природе, которые научат человека управлять этими самыми силами. Неслучайно в это время получают широкое распространение учение герметизма (труды, надписанные именем Гермеса, а ренессанс сильно уважает древность), который учит, что человек, будучи образом Бога, имел власть над миром, но эту власть потерял. Но можно вернуть её путём окультных и магических практик. Также популярным становится и пантеизм, но Бога воспринимают, как силу, распределённую по всей Вселенной, которая управляет мировой душой (?). 

Джордано Бруно стал доктором богословия и увлёкся герметизмом и пантеизмом, возрождая культ Солнца (гелеоцентризм), не будучу астрономом и имея о Солнце весьма неверные понятия, поклоняясь ему. Его пытались переубедить 8 лет и стал символом Нового Времени, будучи человеком ренессанса. 1600 год -- год его смерти, стал неким рубежом.

Ренессанс был непродложительным, потому что люди, вдохновлённые стихийным индивидуализмом, попирали любые законы, жизнь была очень тяжёлой, был упадок нравов и доверия, массовое беззаконие. Явной чертой ренессанса был феномен авторства. 
\newpage
\subsection*{2. Философия Нового Времени (XVI век)}
Это эпоха наукоцентризма, главная проблема этой эпохи -- проблема знания, его достоверности, источников его получения.

2 направления, полемизирующих друг с другом:

\begin{table}[]
\begin{tabular}{ll}
Течение     & Источник знания                            \\
Эмпиризм    & опыт; без опыта знания нет                 \\
Рационализм & разум, опыт это ещё не знание (математика)
\end{tabular}
\end{table}

\subsubsection*{1. Эмпиризм Бэкона}
Родоначальник -- Френсис Бэкон (1561 -- 1626). Он принадлежал к династии новой знати в Англии, которая опиралась на низы, был сыном хранителя королевской печати, сделал хорошую карьеру. На склоне лет занялся философией, задумал книгу под названием <<Великое восстановление наук>>, был первым, кто осознал значение знания для государства, у него была знаменитая фраза like <<Знание ведёт к могуществу>>, требует научного подхода, для него это знание о природе. Он критикует схоластику, считает, что она занимается далёкими от реальности вещами. Для него источник знания -- только опыт, он не доверяет разуму.

Есть идолы познания, которые мешают познавать природу:
\begin{itemize}
\item[1)] Идол рода -- разум человека не является надёжным инструментом от дедукции;
\item[2)] Идол пещеры -- индивидуальные особенности человека, которые он получил от воспитания, доверяет предшествущей традиции;
\item[3)] Идол площади(рынка) -- слова, которые придумали люди, но за которыми ничего не стоит (в данном случае, схоластика, в которой множество терминов, за которыми не стоит никакого реального содержания и которые полезны для познания природы)
\item[4)] Идол театра -- Бэкон сравнивает общество со зрителями театра, спектакли в котором -- различные философские учения, в частности учение Аристотеля. Люди воспринимают представления в театре как реальность, но он и может ошибаться. Бэкон выступает с критикой философских авторитетов.	
\end{itemize}

Бэкон считает дедуктивный метод познания неверным, так как разум может опираться на неверные предпосылки, неверную аксиоматику. Он противопоставляет схоластической дедукции метод индукции (от частного к общему) и ипользование эксперимента как средства познания природы [надо допрашивать природу, чтобы выпытать её тайны, в этом желании покорить природу есть некий отголосок ренессанса]. Природа, по его мнению, -- мастерская, которую можно перестраивать под свои нужды.

Эти и другие мысли Френсис Бэкон изложил в незаконченном и коротеньком произведении <<Новая Атлантида>>, которое написано в жанре утопии. В нём путешественники попадают на остров, где правят учёные, в результате кораблекрушения. Бэкон даёт волю своей фантазии, и мы прекрасно можем видеть мировоззрение Нового Времени. Бэкон описывает общество как нечто упорядоченное, организованное. Правители этого острова могут держать все природные явления под своим полным контролем. Эта утопия символизировала чаяния человека Нового Времени.

\subsubsection*{2. Рационализм Декарта}
Другим важным представителем философии данной эпохи был Рене Декарт (1596 -- 1650). Он родился в аристократической семье и обучение прошёл в колледже Ла Флеш, колледже, учреждённом орденом иезуитов, которые проводили политику создания элитных учебных заведений, чтобы потом через элиту реализовывать свои цели. 

Рене Декарта интересовал вопрос, насколько мы можем доверять своим знаниям, ведь Декарт жил в эпоху, последующую за Ренессансам, и скептически относился к наукам Ренессанса. Его интересовала теория познания, методы познания, которые гарантируют достоверность результата. 

В 1637 году выходит его труд под названием <<Рассуждения о методе>>, где Декарт формулирует 4 правила метода познания:
\begin{itemize}
\item[1)] принимать только ясное и отчётливое для нашего ума;
\item[2)] разлагать сложное на части;
\item[3)] восходить от простого к сложному -- дедукция (<<Начала>> Евклида);
\item[4)] делать полные обзоры -- не делать скачков в знаниях, методично охватывая все необходимые вопросы.  
\end{itemize}	

Декарт ведёт себя, как математик. Но каковы же должны быть аксиомы, на которых должно быть выстроено знания?	А остальное выводится из них. Решению задачи нахождения таких истин в первой инстанции была посвящена его книгда <<Размышления о первой философии>>, вышедшая в 1641 году. Он считает, что знания надо строить на новом фундаменте достоверных первоначальных истин. Эти размышления построены так, что мы видим не только сами результаты выводов Декарта, но и путь, который он проделал, чтобы к ним прийти. Это произвдение было опубликовано вместе с критическими замечаниями коллег философа и ответами на них.

Всего размышлений шесть.
\begin{itemize}
\item[Первое размышление] говорит о том, что для нахождения достоверных истин необходимо всё поставить под сомнение. Казалось бы, что единственная наука, в которой всё просто, -- это математика, но мы можем быть неправы в собственном сознании.
\item[Второе размышление] указывает: мы можем сомневаться в содержании наших мыслей, но не можем сомневаться в факте нашего сомнения, это для нас очевидно. Сомнение -- это действие мысли, значит то, что наша мысль существует -- самоочевидно:
\begin{center}
Cogito, egro sum
\end{center}
Наши мысли [идеи (не как у Платона) существуют в нашем уме] могут быть основ	аны на чувственном восприятии. К примеру, информацию о Солнце или огне мы получили посредством тела. Есть мысли, которые являются плодом нашего воображения, как Минотавр или пегас. Но Декарт выделяет и третий тип мыслей: мысли врождённые, которые внутри нас, но мы не могли их подчерпнуть откуда-то ещё, например, идея истины. 

Далее Декарт переходит от идей к онтологии. С его точки зрения, сам факт сомнения говорит о том, что человек не совершенен, потому что совершенное не сомневается. А совершенство своё можно осознавать, потому что в уме уже есть идея совершенства, которая является неким эталоном осознания совершенства. И Декарт говорит, что одной из врождённых идей является идея Бога -- совершенного бесконечного, ничем не ограниченного существа. По Декарту конечное не может вообразить бесконечного, поэтому идея бесконечности врождённая, идея Бога указывает на существование Бога -- без реального Бога у нас бы не было его идеи.

Декарт поднимает вопрос о первопричине идей в нашем сознании. Идея отражает реальность. Если бы не было Бога, не было бы идеи Бога. Он называет Бога <<бесконечной субстанцией>>, субстанция -- то, что не нуждается в другой вещи для своего сущестования. Также Декарт различает ещё два вида конечных субстанций (res) (нуждаются в содействии Бога):

\begin{itemize}
\item res cogitans -- вещь мыслящая (духовное) -- наш ум, наше Я, обладает возможностью мыслить, желать, рассуждать;
\item res extansa -- вещь протяжённая (материальное) -- протяжённость: всё материальное занимает место в пространстве.
\end{itemize} 

Рене Декарт формулирует проблему психо-физического дуализма. Наше сознание имеет дело только со своими представлениями, даже когда изучает окружающий мир, тела, наши мысли об этих телах такого не имеют. Сознание работает только со своими идеями. Отсюда вопрос: как же у человека соединены мыслящий ум и тело, ведь они совершенно разной природы. Тело взаимодействует с другими телами, а мысли -- с другими мыслями, чувствами, представлениями. И между ними существует грань, ряд тел и ряд мыслей параллельны, сосуществуют вместе.
\end{itemize}

Декарт далее выстраивает свою философию. С точки зрения Декарта, задача этики состоит в том, чтобы человек научился управлять собой, своим телом (автоматом, которым управляют). С точки зрения Декарта, наш ум -- надёжный инструмент познания, мы можем ему доверять, потому что ум дан совершенным Богом. 

Ошибки ума возникают из-за воли, мы выдаём желательное за действительное, совершая ошибки в рассуждениях. Нужно понять разницу между добром и злом -- путь последовательного рационализма.

\subsubsection*{3. Блез Паскаль}

Среди современников Декарта был человек, который предложил философскую концепцию, которая оказалась востребованной в том числе и в ХХ веке, -- Блез Паскаль (1623 -- 1662). Его отец, Этьен Паскаль, будучи чиновником, хотел дать детям хорошее образование, даже разработал план воспитания детей. 

Что касается мировозрения Блеза, всё началось с появления в их доме янсенистов -- антагонистов иезуитов, которые оставили ему их главный их труд <<О преобразовании внутреннего человека>>. Человек находится под влиянием трёх желаний: чувственные (пьянство и прочее), познание (желание использовать чувства для испытания неизвестного, второстепенное выбирается вместо главного), желание возвыситься над другими (хозяин самому себе и другим людям, быть подобием Бога, целью будет именно самовозвышение).

После изобретения счётной машины, которую показали королю, стал известен (в XVII веке была уже мода на науку). Это знакомство со светским обществом подвинуло Паскаля к разными философским размышлениям, он хотел бы изучить человека. Паскаль, наблюдая за свестским обществом, порывает с ней из-за её искусственности, он уходит с этой поверхности жизни, пережая в Пор-Рояль (центр янсенистов), где вступает в полемику с иезуитами, пишет <<Письма к провинциалу [должность в иезуитской иерархии]>>, где выступает с критикой иезуитской казуистики (от слова <<слуйчай>> по-латыни). Иезуиты старались построить мораль, где цель оправдывает средство. Паскаль показывает на реальных примерах, что это приводит к тому, что человек оправдывает любой свой поступок, любое своё желание. Эта книга произвела фурор во французском обществе. 

Самое известное произвдение Паскаля, незаконченное, известно под названием <<Мысли>>. Оно становится интересно позже. В нём Паскаль излагает своё видение философской антропологии. Начинается оно с критики рационализма: когда мы думаем, что с помощью своего разума можем познать мир, наш разум находится между двумя бесконечностями: бесконечностью Вселенной и бесконечностью микромира (бесконечностью в малом). Мы познаём только какие-то части, но связать мы их можем, только зная целое. Поэтому когда мы строим новые знания, фундамент может оказаться шатким, ведь нельзя быть уверенным в надёжности этих знаний. Кроме того, познавая мир, мы должны иметь в виду ограниченность наших чувств и то, что мы не понимаем, что мы сами такое. Также наш ум -- руководитель нашей жизни -- он, как флюгер, дует туда, куда дует ветер (наша воля, наши желания). На ум действуют обманывающие силы, заблуждаясь в истинности наших мыслей из-за воображения [неправильно оцениваем важность чего-либо], самолюбия [мы должны благодарить людей, которые указыают нам на наши недостатки], тщеславия [многие вещи делаем напоказ, приукрашивая] и привычки [о многих вещах судим по собственным привычкам].

При этом Паскаль говорит о том, что нас окружает Вселенная, но она ведь тоже не знает, что она существует, но человек знает и ощущает себя в этом. Совершенство мысли заключается в правильно направленной воле человека, которую он связывает с сердцем, которым мы принимаем самые важные вещи, которым мы любим и ненавидим, -- центром человека. Человек сердцем может познать самые важные истины. Если он ими руководствуется, то он мудрее других людей.

Паскаль делит людей на три категории:
\begin{itemize}
\item люди простые необразованные, но если своим сердцем они восприняли важные истины, они мудрее многих наполовину разобравшихся людей;
\item великие умы, желающие познать всё, но видящие при этом всё своё невежество, начинают руководствоваться теми же истинами, что и простые люди.
\end{itemize}
Первая и третья категории поддерживают течение жизни.

Ещё открылось Паскалю, что человек без разума не человек. Это величие становится таковым, когда человек видит свою ограниченность. Человек -- загадочное существо: его величие разума и мудрости может сочетаться с клоакой разных пороков, в нём может сочетаться жажда счастье и желание жаловаться, они несчастливы. Значит, человек чувствует, что он предназначен для другого. Нужно разобраться в себе, чтобы сменить путь на правильный, но человек увлекается суетой, хочет занять себя процессом, а не результатом -- процесс помогает увлечь себя, помогает забыться. Человек не хочет оставаться наедине с самим собой, чтобы не думать о вечном, но умирать каждому придётся в одиночку. 

Паскаль предполагает, что когда человек поймёт, что высшее проявление разума -- понять, что есть вещи, превосходящие твоё понимание, он начнёт искать опоры, он может познать и открыть рай и ад этой жизни.

Паскаль предлагал знакомым, для которых разработал первый теорвер (для ставок), <<пари Паскаля>>: они признают, что есть бессмертие и Бог, другие -- нет. На что ставить? Если первая ставка срабоает, то выигравшие получают всё, а вторые всё теряют, а если вторая, то никто ничего не получает. С точки зрения Паскаля, разумно выбрать первую ставку -- божественную мудрость надо полюбить, чтобы познать. Он был очень религиозным человеком, но его теория, чем-то напоминавшая эксистенциализм, нашла своих приверженцев позже.

\section*{11. Продолжение английского эмпиризма. Рационализм}
\subsection*{1. Томас Гоббс}
Томас Гоббс (1588 -- 1679) -- продолжатель английского эмпиризма. Жил во времена революции в Англии -- строил социальную философию. Он считал, что причина этой войны есть незнание обществом, что такое война и мир, каким должно быть общество.

Но в духе философии своего времени он говорит сначала о мире. В своём основном труде <<Основы философии>> он в первом томе [онтология] пишет о теле, во втором [гносеология] -- о человеке, в третьем [социальная философия]-- о гражданине. В обществе же Гоббс известен другим своим произвдением -- о том, каким должно быть государство: <<Левиафан>> (1651).

Если говорить об онтологии Гоббса, то он говорит, что существуют только тела за одним исключением Бога, о котором, кроме того, что он точно есть, больше ничего сказать не можем. Тела взаимодействуют путём удара. Гносеология Гоббса -- сенсуализм (посредством впечатлений) и номинализм (наше мышление -- это операции со знаками). Отсюда истина (свойство высказываний) -- непротиворечивое употребление знаками, значения слов возникают по договорённости (как?)

По Гоббсу, есть познания априори (заранее постулируем) и апостериори (из опыта).

С точки зрения Гоббса, люди равны, они созданы равными Богом, и Гоббс считает, что в естественном состоянии действует естественный закон: сохранить сво жизнь и удовлетворить свои естественные потребности, который выражается в ествественном праве -- делать всё для выполнения этого закона. 

Но с точки зрения Гоббса, такая ситуация естественного состояния приводит к войне всех против всех, потому что будет соперничество, будет равенство надежд на достижение тех или иных целей,  но из-за стремления к самосохранению люди приходят к понятию общественного договора, когда люди договариваются ограничить свои естественные права, формируя государаства. Граждане государства отказываются от части своих прав, передавая их соверену (главе государства), и уже при появлении государства вместо естественного закона, начинает действовать тот общественный договор (закон государства).

Сам же Гоббс считал, что наилучшей формой управления государством является абсолютная Монархия. Гоббс пишет, что монарх отвечает только перед Богом, творцом закона, и больше не перед кем другим. Гоббс, описывая в <<Левиафане>> права и обязанности соверена, подчёркивает плюсы такой формы правления. Глава должен руководить армией, законодательной и исполнительной власти, издавать законы и так далее. <<Неудобство такой власти хуже безвластия>>, --  пишет Гоббс. Он проводит сравнение: общие интересы выигрывают там боле всего, где они теснее связаны с частными интересами, а в случае монарха благополучие страны непосредственно связано с благополучием монарха (в этом же и плюсы наследственной монархии). Кроме того, если в процессе участвует парламент, то они могут не сохранить тайну, затянуть принятие важного решения или использовать свои личные интересы против благополучия государства.

Примечательно, что у Гоббса вопросы о том, что считать добром и злом, что правильным, что -- нет, являются результатом некого договора. 

\subsection*{2. Джон Локк}
Джон Локк (1632 -- 1704 гг) активно участвовал в политической жизни Англии, закончил Оксфордский колледж, был врачом лорда. При реставрации династии Стюартов имигрировал во Францию вместе со своим патроном, но вернулся при Бархатной революции, установившей конституционную монархию, идеологом которой по сути и являлся.	

\subsubsection*{1. Гносеология Локка}
Основной его труд -- <<Опыт о человеческом разуме>>(примерно 1671 - 1691). В этом труде Локк попытался в виде некоторой системы представить доктрину эмпиризма. С точки зрения Локка, человек при рождении представляет собой tabula rasa -- чистую доску -- и только через окна и двери органов чувств получает какие-то представления об окружающем мире -- последовательный эмпиризм. Его современники эти представлеия называют <<идеями>>, их можно получить:
\begin{itemize}
\item из ощущений -- внешний опыт [цвет и т.д.];
\item из рефлексии -- внутренний опыт [из процесса наблюдения за своими собственными мыслями (формируется воля, мышление и т.п.)]
\end{itemize}

Локк критикует концепцию врождённых идей Декарта -- даже математические идеи и идеи существования Бога формируются на основе опыта.

Локк делит идеи на простые и сложные. К простым, например, относятся идеи цвета и протяжённости, а к сложным -- те, что формируются нашим умом путём операций сравнения и абстракции. Идея субстанции -- тоже абстракция: в опыте мы имеем дело с качествами вещей, но то, что вещи имеют какие-то носители -- субстанции, -- мы постигаем путём операций ума.
\newpage
Джон Локк делит качества, постигаемые на опыте, на 2 вида:
\begin{itemize}
\item[первичные:] качества, присущие вещам независимо от наших восприятий (протяжённость, движение)
Примечание: во времена Локка была распространена корпускулярная теория, где предполагалось, что многие динамические характеристики частиц существуют независимо от нас.
\item[вторичные:] есть только в субъекте восприятия, но не присущи самой вещи (цвет, вкус) 
\end{itemize}

\subsubsection*{2. Социальная философия}
Ещё в 1679 году Гоббс издаёт <<Два трактата о правлении>>, где излагает свою социальную философию. Он тоже признаёт наличие естественного состояния и существование естественных прав, но вносит важную коррективу, отмечая, что у Гоббса, что неправильно то, что людьми, кроме стремления к самосохранению, ничего не руководило. Локк отмечает, что если естественный закон, что дан Богом, то он не может быть направлен во вред людям, он скорее нравственный. 

Гоббс был сторонником механистической теории, корпускулярной теории, воспринимая человека как атом [социальный атомизм], который прибегает к взаимодействию с другими людьми только для удовлетворения каких-то личных потребностей. Локк же смягчает эту теорию: нравственный закон, вложенный Богом в природу человека, является совестью и выражается в понятии здравого разума. На аргумент, почему не все нравственны, Локк отвечает, что не все полльзуются им в полной мере, нравственности также мешают и привычки, неправильно сформированные в ходе воспитания.

Законы государства по Локку должны же опираться на этот нравственный закон. И отсюда он делает вывод, что не было войны всех против всех, такк этот закон действовал и в естестенном состоянии. Также Локк признаёт, что общество -- более высокий уровень, чем естественное состояние, потому что опирается не только на нравственный закон, а человек в своём отпадшем от Бога состоянии не способен дейстовать по нравственному закону, ему нужны внешние законы государства, чтобы ограничивать его.

В отличие от Гоббса, Локк признаёт верхновенство конституционной монархии -- закона как верхновной инстанции в государстве и развивает идею разделения власти на 3 ветви: законодательную, исполнительную и судебную, каждая из которых автономна.

У Локка центральное место занимает понятие собственности, связанное с понятием ествественного права. Человек по естеству является владельцем себя, своей души и тела, и если он, занимаясь проиводством чего-то, вложил свои силы и время, душу и так далее, то он является владельцем этой вещи, ибо вложил в неё часть себя. Отношения собственности в государстве регулируются общественной договорённостью, которая заключается в охране права на эту собственность.

Социальная концепция Нового времени государство ориентировано на то, чтобы обеспечить телесную жизнь человека. Если в античный период задача государства -- сделать человека нравственным, то у Гоббса и Локка государство утилитарно -- инструмент регулирования отношения между людьми, но без государства жить можно.	Это связано с тем, что если античность и особенно средневековье рассматривало жизнь человека как временный этап: человек себя готовит к более совершенной жизни, у которой никогда не будет завершения, то в Новое время единственной ценностью становится земная жизнь и благополучие в этой жизни, соотвественно, от государства более ничего не требутеся.

\subsection*{3. Бенедикт Спиноза}
Бенедикт Спиноза (1632 --  1677) был родом из еврейской семьи, жил в Голландии, был отлучён от своей религиозной общины.

Он оставил после себя большое произвдение <<Этика>>(1675). Оно написано по аналогии с <<Началами>> Евклида, как аксиоматизированая дедуктивная система. 

Спиноза начинает с того, что даёт определение субстанции -- это causa sui -- <<причина самой себя>>, зависит только от себя и существует с необходимостью. И особенность его философии в том, что эта субстанция единственна -- это Бог или природа. У субстанции есть различные modus (конечные и бесконечные состояния субстанции) -- то, как она нам представляется в уме. Главные два атрибута -- протяжённость и мышление (в отличие от Декарта). Субстанция бесконечна. Бог не отделим от природы. 

Спиноза использует понятие свободы, определяя его так: <<Свободно только то, что само определяет себя к действию и действует только с необходимостью в силу своей собственной природы,>> -- свободна только божественная субстанция, modus же не существует с абсолютной необходимостью. 

Спиноза не считает эмпиризм чем-то серьёзным, ибо в чувствах субъект смешан с объектом, ничего нельзя разобрать, поэтому единственный источник знаний -- разум. 

Этика Спинозы в том, чтобы осознать причины всего, а философия в этом смысле просвящает человека, а человек мыслит с необходимостью -- свободна с одной стороны отрицается, а с другой -- предполагается. Существование Бога более очевидно, чем существование всего остального.

\subsection*{4. Вильгельм Фридрих Лейбниц}
Вильгельм Фридрих Лейбниц(1646 -- 1716 гг), родился в Лейпциге, учился в Юрицидическом институте, занимался также и математикой и логикой, был вообще очень разносторонним человеком.

Он считал, что историю философии забывать нельзя, необходим некий синтез того, что было изучено в прошлом с тем, что разрабатывается в настоящем.
Основные труды были немногочисленны в силу занятости:
\begin{itemize}
\item <<Новые опыты о человевеческом разуме>>(1704) -- мягкая полемика с Локком (в сторону рационализма);
\item <<Теодицея>>(1710) -- апология Бога, ответ на проблему зла (Бог либо всемогущ, но не благ, и наоборот)
\item <<Монадология>>(1714) -- программное компактное изложение философских взглядов.
\end{itemize}  

\subsubsection*{1. Гносеология}
Лейбниц согласен с тем, что чувства -- источник знаний для человека.
Он подчёркивает важность опыта, но все ли истины зависят от опыта? Среди знаний, которые преобретает человек, он выделяет истины опытные, или истины факта [индукция + примеры]. Чувства недостаточны для того, чтобы установить необходимость истины, дать общие знания. Пример -- смена дня и ночи, состояние Земли и Солнца меняется, поэтому нельзя всегда доверять чувствам. ($\exists x: s(x) \wedge p(x)$, нет импликации -- нет необходиомости )

Аксиомы и правила выбора дают истины разума ($\forall x: s(x) \rightarrow p(x)$) -- логику, мораль, метафизику, математику нельзя вывести из факта.

Лейбниц формирует основные законы логики (тождественности, непротиворечия). Он видит задачу в анализе идей, с помощью него выявить непротиворечивость, выступает за формализацию знания.

Лейбниц делил все представления на ясные (когда можем найти признак, чтобы отделить предмет от других) и тёмные. Идеал знания -- ничего неясного не остаётся, часто прибегая к символам. Лейбниц говорил, что наше сознание больше того, что мы часто не понимаем.

\subsubsection*{2. Онтология}
Лейбниц различает два мира.
\begin{itemize}
\item мир реальный (дискретный, изучает метафизика, состоит из монад (субстанций), неделимых и действующих или испытывающих воздействие других) каждая монада -- зеркало универсума, имеет дело только со своими состояниями, чем совершннее, тем отчётливее; 
\item мир явления (феноминальный) (мир опыта в пространстве и времени -- мир отношения между явлениями, здесь работает матанализ и физика)
\end{itemize}	

Отношения Бога с разумными монадами -- отношения законодателя с государством, это лучший из миров.

Лейбниц выдвигает учение о предустановленной гармонии: как законы механики и нравственные законы сосуществуют? По мнени Лейбница, всё Бог продумал.

По поводу веры и разума, люди путают разум и то, что для них привычно; разум -- это цепь рассуждений, которые истинные и необходимые, вера не противоречит разуму, только представлениям отдельного человека.

Три вида зла:
\begin{itemize}
\item зло метафизическое -- в силу ограниченности;
\item зло моральное -- нравственное (самое опасное);
\item зло физическое.
\end{itemize}
Бог причиняет человеку физическое зло, чтобы оградить от морального.

\addtocontents{toc}{\protect\setcounter{tocdepth}{1}}
\addtocontents{toc}{\protect\setcounter{tocdepth}{2}}
\addtocontents{toc}{\protect\setcounter{tocdepth}{3}}
\tableofcontents 
\end{document}