%Entry
\documentclass[a4paper, 12pt]{book} %type of the doccument: report, book;

%Russian language
\usepackage[T2A]{fontenc} %code 
\usepackage[utf8]{inputenc} %code of original text
\usepackage[english, russian] {babel} %localisation and transfers

%Images 
\usepackage{graphicx}

%Math pack
\usepackage{amsmath, amsfonts, amssymb, amsthm, mathtools}

% Header
\title {Лекции по философии. 8 семестр. Преподаватель: В.И. Коцюба}

\begin{document}
\maketitle
\newpage
Предмет, который мы будем изучать в этом семетре, -- философия. 

\section*{0. Ознакомительная лекция}
\subsection*{1. О термине и нуждах}
Все Вы наверняка слышали слово "Философия", оно греческое и происходит от двух слов: $\varphi \iota \lambda \varepsilon \omega $ -- "любить, иметь приязнь к чему-либо" и
 $\sigma o \varphi \iota \alpha$ -- "мудрость".
 
Первый текст в истории, где встречатеся этот  термин, -- трактат по медицине из Гиппократовского корпуса, датируемый примерно 450-420 гг. до н.э. Этот текст говорит о том, что в 5-ом веке до н.э. термин "философия" уже был привычным в греческой культуре, чего не нужно было пояснять. Из контекста его применения становится понятно, что философы изучали, что такое человек, писали что-то о природе. 

Обратимся ко второй части слова "философия": $\sigma o \varphi \iota \alpha$. Как же понимали эту "софию"? Если мы посмотрим на самую древнюю эпоху (11-9 века до н.э. -- "тёмные века" истории Древней Греции), когда были написаны произвдения Гомера. У него мудрость понимается как "сноровка, умение что-то делать". Но уже в более позднюю эпоху под мудростью начинают понимать и любознательность в широком смысле, и жизненный опыт.

Далее в греческой традиции возникает термин "Пифагорейство". Древние Греки считали, что понятие "философии" ввёл Пифагор, который утверждал, что мудр только Бог, а человек может считаться только любителем мудрости, потому что человек не может изострить свой ум настолько, чтобы считать себя мудрым, стать Богом. 

К настоящему времени философия изучается как некая дисциплина, но что же она изучает, каков её предмет? В разные времена под мудростью понимали разные вещи, была некая неоднозначность того, что значит любить мудрость. Обратившись к первым текстом, с мудростью было связано не только умение познать мир, но также и умение, как жить в правильно в этом мире; философия неразрывно связана с этикой. Впоследствие, во времена Платона и Аристотеля, философия приобретет конкретное определение (Аристотель): предмет философии в том, что она изучает сущее в целом. Если другие науки изучают какие конкретные аспекты реальности, то задача философии как науки -- познать реальность в целом, в общем, как человек познаёт реальность. Также, с точки зрения Аристотеля, важно познать, от чего зависит реальность ("начала реальности"), познать в чём заключается \textit{благо} -- какова конечная цель наших действий (позже: в чём смысл жизни). Те или иные знания -- это средства для достижения смысла жизни. 

В одном из "Диалогов" Платона описывается, как Сократ встретился к Алкивиадом. Сократ, узнав, что Алкивиад пошёл просить у Богов исполнение своего желания, задался вопросом, принесёт ли пользу такая просьба (Алкивиад не задумывался об этом). Ведь человек может достичь желаемого и пожалеть об этом -- если чего-то желаешь, надо задуматься, чтобы не жалеть об исполнении. Существует ли что-то такое в жизни, что исполнится и об этом не пожалеешь? Сократ привёл такой жизненный пример человека, слишком жаждавшего власти, но убитого своими соракниками, после того, как достиг желаемого. Что же нашёл Сократ? Алкивиад решил, что это знание -- особый вид богатства. Знаний -- это благо. Но какое из знаний является наиболее важным для человека? С точки зрения Сократа, знания -- это средства достижения наших целей (всего лишь). Если в порыве гнева один человек исхитрится и помощью имеющихся знаний убьёт другого, то он будет об этом жалеть, что нарушает ценность блага. Отсюда вывод: \textit{знания полезны только тогда, когда мы их направляем для достижения правильных целей}. Самое важное знание, с точки зрения Сократа, -- это умение различать добро и зло; ошибочность в этом вопросе привлекает различные проблемы в нашу жизнь.

Рассмотрим историю философия, её параллели с развитием человеческой истории.

С точки зрения лектора, предметом философии в современном понимании является мировоззрение в широком смысле, система тех норм, ориентиров, идеалов, которыми человек руководствуется по жизни. Каждый философ ставит перед собой свою систему ценностей. Сократ стремился сформировать то, что мы сейчас называем \textit{объективным знанием}, претендуя на универсальность, обосновывая свою позицию. Платон говорил: "Философ -- тот, кто любит истину". Филосфия аккумуилирует опыт и знания, умение жить, выстраивать её распорядок. Также жизнь формируется традицией, например, образовательной -- опыт, полученный философией, получается полезным и в современной жизни.

\subsection*{2. Терминология разделов философии}
\begin{itemize}
\item Онтология.
Этот термин происходит от греческого $o \nu$ -- "сущее". Этот раздел даёт учение о реальности в целом, как она устроена. Философия рассматривает реальность, включая того, кто её познаёт.
 
В философии пригодилось понятие абсолюта. Absolutu (лат.) -- "от всего оторвано, ни от чего не зависит". Всё другое зависит от него, абсолют зависит от самого себя и  только. Философы рассуждают о том, что является тем, от чего зависит всё остальное в мире.

\item Гносеология. Образовано от греческого слова "знание", этот раздел изучает источники, формы и критерии достоверного знания. Началось всё с античности. 

\item Этика. Образовано от греческого слова $\eta \theta o \zeta$ -- "нрав, характер, образ жизни". Этика призвана дать критерии различения добра и зла и правила жизни, как достичь блага.

\item Антропология. Философская антропология (учитывая знания биологии) изучает место человека в реальности и человеческую природу. Homo sapiens -- человек разумный, но в чём проявлется разумность, которая диктует цели, в том числе и достижение совершенства, но что оно из себя представляет?

\item Эстетика. Наука о прекрасном, рассматривает вопрос о критериях красоты, ведь красота зачастую неявно рассматривается как совершенство: ничего ни добавить, ни отнять. Анализ красоты как проявления совершенства, анализа художественного творчества -- предметы эстетики. Художественное творчество -- это выражение мировоззрения, с точки зрения эстетики. 	

Среди "маленьких трагедий" А.С.Пушкина есть очень интересный сюжет "Моцарт и Сальери" о гении и посредственности. Но не всё так гладко! "Вот, говорят, нет правды на Земле,  но правды нет и выше", -- возмущается Сальери. Сальери возмущается тем, что он сделал всю свою жизнь "подножием искусству", но Моцарт хочет жить и простой жизнью, что по мнению Сальери, есть предательство. Вот он всё и исправил. Проблема гениальности и простого служения деньгам -- большая в творчестве Пушкина. Таким образом, художественные произвдения выражают мировоззрение автора.

Ещё один пример -- что выражает человека объективнее, фотография или портрет. Художник рисует человека, как он его видит, каким хочет его видеть, а фотография не предлагает такого участия. Но, с другой стороны, фотография фиксирует миг, мгновенное неинвариантное состояние человека. Портрет объективнее в этом смысле, что на нём изображается внутренний мир человека через внутренний мир художника. Понимание человека проходит через понимание его внутреннего мира. Недаром у Аристотеля поэзия ближе к философии, чем история, потому что история фиксирует единичное событие, а поэзия -- некий закон, правило реальности.

\item Социальная философия. Она ставит вопрос, каким должно быть общество, какие нормы должны быть в жизни, а социология делает выводы о том, каким оно является, отображая картину реальности. 

\item Философия истории, которая рассматривает тему исторического процесса, его направленность. В 18-м веке философы разработали идея прогресса, делая аналогию истории с взрослением человека. Она ставит вопрос о смысле жизни человека, его перспективах.
 	 
\end{itemize}
\textit{Философия ставит цель формирования цельной картины жизни, интегрирующей функции знания.}

\subsection*{3. Античная философия}

Прблема Нидема -- английский учёный, изучавший китайскую культуру, сформулировал такой вопрос: "Почему Китай, имевший высокое культурное развитие, не получил толчка промышленности, как в Европе?" 

Почему античная философия по большей части греческая? По мнению лектора, это может быть связано с тем, что в мировоззрении древних греков религия имела мало нравственных компонентов формирования мировоззрения, в отличие от того же древнего Египта. Боги -- не образцы для людей. Потребность в таком мировоззрении, потребность в таком знании послужила предпосылками к возникновению нормативной философии, дистанцирующей себя от религии.

\subsubsection*{Периодизация греческой философии}
\begin{itemize}
\item[*] Фалес, 6-ой -- 5-ый вв. до н.э.-- период "досократики". Общая черта этого периода: философы интересовались, как устроена природа, откуда всё пошло, поэтому их называют ещё "натурфилософы". По их мнению, "разум призван руководить"

\item[*] Сократ, 5-ый -- 4-ый вв. до н. э.  -- "Классический" (образцовый) период. Появляются 2 очень значимых философа: Платон и Аристотель (по преемственности). Дальнейшая философия -- это некая разработка вопросов, сформулированных произведениями Платона. Аристотель -- это "отец" почти всей науки: физики, этики и т.п. Эти школы пройдут через столетия.

\item[*] 330 -- 30 гг. до н. э. -- Эллинизм -- эпоха, когда благодаря завоеваниям Александра Македонского греческая культура распространилась на огромные территории, греческий язык стал фактически международным, греческая культрура стала базисом взаимодействия других культур. Символ этого периода -- город Александрия.

\item[*] 1-ый век до н.э. -- 6-ый век н.э. -- Позднеантичная, или греко-римская философия. Наступает расцвет комментаторской традиции: если ты умный, начинай понимать и комментировать тексты тех, кто писал до тебя. Типичная техника обучения в те времена: слушали и комментировали прочитанное. 
\end{itemize}

 




\end{document}