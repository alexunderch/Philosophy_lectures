%Entry
\documentclass[a4paper, 12pt]{book} %type of the doccument: report, book;

%Russian language
\usepackage[T2A]{fontenc} %code 
\usepackage[utf8]{inputenc} %code of original text
\usepackage[english, russian] {babel} %localisation and transfers

%Images 
\usepackage{graphicx}

%Math pack
\usepackage{amsmath, amsfonts, amssymb, amsthm, mathtools}

% Header
\title {Лекции по философии. 8 семестр. Преподаватель: В.И. Коцюба}

\begin{document}
\maketitle
\newpage
Предмет, который мы будем изучать в этом семетре, -- философия. 

\section*{0. Ознакомительная лекция}
\subsection*{1. О термине и нуждах}
Все Вы наверняка слышали слово "Философия", оно греческое и происходит от двух слов: $\varphi \iota \lambda \varepsilon \omega $ -- "любить, иметь приязнь к чему-либо" и
 $\sigma o \varphi \iota \alpha$ -- "мудрость".
 
Первый текст в истории, где встречатеся этот  термин, -- трактат по медицине из Гиппократовского корпуса, датируемый примерно 450-420 гг. до н.э. Этот текст говорит о том, что в 5-ом веке до н.э. термин "философия" уже был привычным в греческой культуре, чего не нужно было пояснять. Из контекста его применения становится понятно, что философы изучали, что такое человек, писали что-то о природе. 

Обратимся ко второй части слова "философия": $\sigma o \varphi \iota \alpha$. Как же понимали эту "софию"? Если мы посмотрим на самую древнюю эпоху (11-9 века до н.э. -- "тёмные века" истории Древней Греции), когда были написаны произвдения Гомера. У него мудрость понимается как "сноровка, умение что-то делать". Но уже в более позднюю эпоху под мудростью начинают понимать и любознательность в широком смысле, и жизненный опыт.

Далее в греческой традиции возникает термин "Пифагорейство". Древние Греки считали, что понятие "философии" ввёл Пифагор, который утверждал, что мудр только Бог, а человек может считаться только любителем мудрости, потому что человек не может изострить свой ум настолько, чтобы считать себя мудрым, стать Богом. 

К настоящему времени философия изучается как некая дисциплина, но что же она изучает, каков её предмет? В разные времена под мудростью понимали разные вещи, была некая неоднозначность того, что значит любить мудрость. Обратившись к первым текстом, с мудростью было связано не только умение познать мир, но также и умение, как жить в правильно в этом мире; философия неразрывно связана с этикой. Впоследствие, во времена Платона и Аристотеля, философия приобретет конкретное определение (Аристотель): предмет философии в том, что она изучает сущее в целом. Если другие науки изучают какие конкретные аспекты реальности, то задача философии как науки -- познать реальность в целом, в общем, как человек познаёт реальность. Также, с точки зрения Аристотеля, важно познать, от чего зависит реальность ("начала реальности"), познать в чём заключается \textit{благо} -- какова конечная цель наших действий (позже: в чём смысл жизни). Те или иные знания -- это средства для достижения смысла жизни. 

В одном из "Диалогов" Платона описывается, как Сократ встретился к Алкивиадом. Сократ, узнав, что Алкивиад пошёл просить у Богов исполнение своего желания, задался вопросом, принесёт ли пользу такая просьба (Алкивиад не задумывался об этом). Ведь человек может достичь желаемого и пожалеть об этом -- если чего-то желаешь, надо задуматься, чтобы не жалеть об исполнении. Существует ли что-то такое в жизни, что исполнится и об этом не пожалеешь? Сократ привёл такой жизненный пример человека, слишком жаждавшего власти, но убитого своими соракниками, после того, как достиг желаемого. Что же нашёл Сократ? Алкивиад решил, что это знание -- особый вид богатства. Знаний -- это благо. Но какое из знаний является наиболее важным для человека? С точки зрения Сократа, знания -- это средства достижения наших целей (всего лишь). Если в порыве гнева один человек исхитрится и помощью имеющихся знаний убьёт другого, то он будет об этом жалеть, что нарушает ценность блага. Отсюда вывод: \textit{знания полезны только тогда, когда мы их направляем для достижения правильных целей}. Самое важное знание, с точки зрения Сократа, -- это умение различать добро и зло; ошибочность в этом вопросе привлекает различные проблемы в нашу жизнь.

Рассмотрим историю философия, её параллели с развитием человеческой истории.

С точки зрения лектора, предметом философии в современном понимании является мировоззрение в широком смысле, система тех норм, ориентиров, идеалов, которыми человек руководствуется по жизни. Каждый философ ставит перед собой свою систему ценностей. Сократ стремился сформировать то, что мы сейчас называем \textit{объективным знанием}, претендуя на универсальность, обосновывая свою позицию. Платон говорил: "Философ -- тот, кто любит истину". Филосфия аккумуилирует опыт и знания, умение жить, выстраивать её распорядок. Также жизнь формируется традицией, например, образовательной -- опыт, полученный философией, получается полезным и в современной жизни.

\subsection*{2. Терминология разделов философии}
\begin{itemize}
\item Онтология.
Этот термин происходит от греческого $o \nu$ -- "сущее". Этот раздел даёт учение о реальности в целом, как она устроена. Философия рассматривает реальность, включая того, кто её познаёт.
 
В философии пригодилось понятие абсолюта. Absolutu (лат.) -- "от всего оторвано, ни от чего не зависит". Всё другое зависит от него, абсолют зависит от самого себя и  только. Философы рассуждают о том, что является тем, от чего зависит всё остальное в мире.

\item Гносеология. Образовано от греческого слова "знание", этот раздел изучает источники, формы и критерии достоверного знания. Началось всё с античности. 

\item Этика. Образовано от греческого слова $\eta \theta o \zeta$ -- "нрав, характер, образ жизни". Этика призвана дать критерии различения добра и зла и правила жизни, как достичь блага.

\item Антропология. Философская антропология (учитывая знания биологии) изучает место человека в реальности и человеческую природу. Homo sapiens -- человек разумный, но в чём проявлется разумность, которая диктует цели, в том числе и достижение совершенства, но что оно из себя представляет?

\item Эстетика. Наука о прекрасном, рассматривает вопрос о критериях красоты, ведь красота зачастую неявно рассматривается как совершенство: ничего ни добавить, ни отнять. Анализ красоты как проявления совершенства, анализа художественного творчества -- предметы эстетики. Художественное творчество -- это выражение мировоззрения, с точки зрения эстетики. 	

Среди "маленьких трагедий" А.С.Пушкина есть очень интересный сюжет "Моцарт и Сальери" о гении и посредственности. Но не всё так гладко! "Вот, говорят, нет правды на Земле,  но правды нет и выше", -- возмущается Сальери. Сальери возмущается тем, что он сделал всю свою жизнь "подножием искусству", но Моцарт хочет жить и простой жизнью, что по мнению Сальери, есть предательство. Вот он всё и исправил. Проблема гениальности и простого служения деньгам -- большая в творчестве Пушкина. Таким образом, художественные произвдения выражают мировоззрение автора.

Ещё один пример -- что выражает человека объективнее, фотография или портрет. Художник рисует человека, как он его видит, каким хочет его видеть, а фотография не предлагает такого участия. Но, с другой стороны, фотография фиксирует миг, мгновенное неинвариантное состояние человека. Портрет объективнее в этом смысле, что на нём изображается внутренний мир человека через внутренний мир художника. Понимание человека проходит через понимание его внутреннего мира. Недаром у Аристотеля поэзия ближе к философии, чем история, потому что история фиксирует единичное событие, а поэзия -- некий закон, правило реальности.

\item Социальная философия. Она ставит вопрос, каким должно быть общество, какие нормы должны быть в жизни, а социология делает выводы о том, каким оно является, отображая картину реальности. 

\item Философия истории, которая рассматривает тему исторического процесса, его направленность. В 18-м веке философы разработали идея прогресса, делая аналогию истории с взрослением человека. Она ставит вопрос о смысле жизни человека, его перспективах.
 	 
\end{itemize}
\textit{Философия ставит цель формирования цельной картины жизни, интегрирующей функции знания.}

\subsection*{3. Античная философия}

Прблема Нидема -- английский учёный, изучавший китайскую культуру, сформулировал такой вопрос: "Почему Китай, имевший высокое культурное развитие, не получил толчка промышленности, как в Европе?" 

Почему античная философия по большей части греческая? По мнению лектора, это может быть связано с тем, что в мировоззрении древних греков религия имела мало нравственных компонентов формирования мировоззрения, в отличие от того же древнего Египта. Боги -- не образцы для людей. Потребность в таком мировоззрении, потребность в таком знании послужила предпосылками к возникновению нормативной философии, дистанцирующей себя от религии.

\subsubsection*{Периодизация греческой философии}
\begin{itemize}
\item[*] Фалес, 6-ой -- 5-ый вв. до н.э.-- период "досократики". Общая черта этого периода: философы интересовались, как устроена природа, откуда всё пошло, поэтому их называют ещё "натурфилософы". По их мнению, "разум призван руководить"

\item[*] Сократ, 5-ый -- 4-ый вв. до н. э.  -- "Классический" (образцовый) период. Появляются 2 очень значимых философа: Платон и Аристотель (по преемственности). Дальнейшая философия -- это некая разработка вопросов, сформулированных произведениями Платона. Аристотель -- это "отец" почти всей науки: физики, этики и т.п. Эти школы пройдут через столетия.

\item[*] 330 -- 30 гг. до н. э. -- Эллинизм -- эпоха, когда благодаря завоеваниям Александра Македонского греческая культура распространилась на огромные территории, греческий язык стал фактически международным, греческая культрура стала базисом взаимодействия других культур. Символ этого периода -- город Александрия.

\item[*] 1-ый век до н.э. -- 6-ый век н.э. -- Позднеантичная, или греко-римская философия. Наступает расцвет комментаторской традиции: если ты умный, начинай понимать и комментировать тексты тех, кто писал до тебя. Типичная техника обучения в те времена: слушали и комментировали прочитанное. 
\end{itemize}

 
\section*{1. Лекция о первом периоде античной философии -- периоде "досократиков"}

\subsection*{1. Этика}
Первым философам-досократикам предшествовали т.н. <<7 мудрецов>>. Их некоторые высказывая сохранились. Основные идеи:

\begin{itemize}
\item Принцип меры 
Один из семи мудрецов, Хилон, выразил этот принцип лаконично: <<Ничего слишком>>. Человеку нужно искать в своих желаниях, поступках и интересах некую меру.

Этот принцип напрямую был применим по потношению к удовольствию. Другой из мудрецов, Клеобул, говорил так: <<Удовольствие обуздывай>>. Или, как говорил, Солон, уважаемый мыслитель, законодатель в Афинах: "Избегай удовольствия, рождающего страдания." Греки обратили внимание на тот факт, что человек, не расчитавший свою скорость, может промахнуться, оступиться -- так и чрезмерное удовольствие рождает старадание.

Мера также проявляет себя через язык, общение с другими людьми. Тот же Клеобул говорил: <<Будь сдержан на язык>>. Цитата Хилона: <<Язык твой пусть не обгоняет ума>>, -- в общении нужна мера, нужно думать, что мы говорим, у каждого разговора должна быть своя разумная цель. Известный поэт Биант Приенский, когда город в котором он жил, претерпел катастрофу, вышел без имущества и объяснил свой поступок фразой: "Всё своё ношу с собой", -- то есть весь мой ум, характер и качества уже и так принадлежат мне. По поводу примера с промахом он и сказал: "Не болтай! Промахнёшься -- пожалеeшь", -- часто в эмоциях человек жалеет о том, что сказал. Мудрость видели в том, что человек умеет следить за своей речью, не сказать лишнего.

Умение управлять собой, обуздать свой гнев также является признаком меры. Солон произнёс: "Научившись подчиняться, научишься и управлять", -- хороший управленец (в том числе и собой) вырастает в умении подчинаяться.

\item Отношения с другими людьми. Общий принцип этих отношений выразил один из 7 мудрецов, Питтак, в <<золотом правиле морали>>, говорящем "что возмущает тебя в другом, того не делай сам". Это правило требует не только того, чтобы следить за другими, но и за своими поступками, речью, делами, -- так мы лучше понимаем принципы отношения и с другими людьми.

Фалес поднял вопрос об \textit{отношении к родителям}: "Какие услуги окажешь родителям, такие в старости ожидают детей".

Вопрос \textit{дружбы} затронул Периандр: <<С друзьями будь одним и тем же, и в удачу, и в беде>>. Как говорил Хилон: "На обеды друзей ходи медленно, а на беды ходи быстро", -- дружба проявлется в том, что оказывается взаимопомощь в беде. Тот же Хилон говорил: "Над попаввшими в беду не смейся".

"Знай себя' , (Хилон) -- из этой фразы следует, что представление человека о себе могут быть ошибочными; античные философы утверждали, что для того чтобы познать себя, необходим первичный самонализ. С точки зрения Фалеса это очень сложно, это кажется очень простым.

\end{itemize}

\subsection*{1. Онтология}

Греческая философия начинается с вопроса, что есть всё, поиска первоэлемента.

Фалес (640 - 546 гг. до н. э.) стал родоначальником Милетской школы. Милет -- город, расположенный на Ионийском побережье Малой Азии. Там появилась первая общепризнанная школа философии.

Примечение: Греки выстраивали свою хронологию по олимпиадам. Период жизни человека обозначался высшей точкой его жизни ($\alpha \kappa \mu \eta$), приходившейся на какую-либо олимпиаду. В древнем Риме на могилах писали эпитафии -- карьерный путь человека.

Фалес считал первоэлементом воду (всё в мире пропитано водой), его ученик Анаксимандр (610 -540 гг. до н.э.) считал первоначалом что-то другое, до сих пор спорят что, но осталась его характеристика, что это начало беспредельное ($\alpha \pi \varepsilon \iota \rho \nu$ ). Вещи на время получают от этого начала свои свойства, но через некоторое время долги нужно вернуть.

Некая аксиома античной онтологии: "Nihil ex nihilo", -- из ничего ничего не бывает. Первые досократики стали различать мир являний, в котором происходит возникнование и исчезновение, за чем существует что-то вечное, что всегда будет. То, что мы видим, -- есть распад и воссоединение этого самого. Это начало можно назвать вечным, ведь увядание красоты вызывало у греков депрессию.

Анаксимандр был одинм из первых прозаиков. Одно из его произвденений -- "О природе" ($\iota \pi \varepsilon \rho \iota  $ $ \varphi \upsilon \sigma \varepsilon \omega \varsigma$). В слове "природа" корень, как и в русском языке, -- "рождать".

Ещё представителем Милетской школы был Анаксимен, который началом всего полагал воздух, он считал, что различные состояния вещества -- это лишь модификации воздуха. 

Далее центр философской жизни перемещается в южную Италию, на Сицилию, где появился центр Пифагорейского движения -- основателем был Пифагор(570 - 497 гг. до н.э.). У учеников этой школы была традиция, они приписывали основателю школы свои произведения.

Пифагорейская школа ищет начало. Она предствляет собой ярко выраженный дуализм, мир объясняется с помощью друх начал, предельного и беспредельного. Предполагают, что Пифагор ввёл термин "космос", от понятия упорядоченной красоты, как противопроставления хаосу. С точки зрения пифагорейцев, космос есть результат взаимодействия этих двух начал; порядок в космосе от предельного начала, которое устанавливает меру, и это красиво. Упорядоченную реальность можно познать. На вопрос, что есть всё, они отвечают: "Всё есть число. Без числа ничто не может быть, без числа ничто нельзя познать". (Филолай). В основе реальности, в которой мы живём, лежат числовые закономерности, то есть число и основа познания реальности.

Пифагорейский союз исследовал чисел, что давало, по их мнению, правильное понимание устройства жизни; нужно находиться в соответствии с миром. Например, в области педагогики они ввели понятие инфантилизма -- отставание умственного развития от телесного -- развитие должно быть гармоничным. Они вводят правило отчёта в конце дня, что в другой день не поддаваться тем же ошибкам, заблуждениям. Пифагорейская школа стала неким образцом для других философских школ.

Среди Досократиков особое место занимает Гераклит (540 - 480 гг. до н.э. плачущий философ). Он плакал, потому что ему было жалко людей, что они зря тратят свою жизнь. Он считал, что люди живут в своих грёзах, когда рассудок у вскх общий. Задача философа -- приобрести этот самый рассудок, "логос" мира (в мире есть некий порядок, некая периодичность), перейти от мнимого к истинному. Ему принадлежит высказывание <<$\pi \alpha \nu \tau \alpha$ $\rho \varepsilon \iota$>>, которое можно перевести как "Всё течёт", изменяется согласно порядку; также мир проникнут некой борьбой, взаимодействием противоложностей; он сравнивал мир с натянутым луком. Первоэлементом Гераклит считал огонь -- Вселенная существует циклически, в определённые моменты рождаясь и сгорая заново; мудрость состоит в том, чтобы познать за многообразием некий общий закон, которому должны подчиняться и люди. Гераклит постоянно критиковал некоторые греческие обычаи, например, пьянство -- поклонение Вакху --, которые были бы нарушением этого закона. Гераклит также известен фразой "Многознание уму не научает", то есть он разделяет знания и мудрость; также он критиковал демократию, считая, что один не тьма, если он наилучший, -- демократия ориентируется на желания толпы, отвергая лучших.
 
Следующей школой греческой философии, которая нам интересна, -- Элейская школа. Она располоагалась в Великой Греции, как и пифагорейская. Первым представителем был Ксенофан -- поэт, его высшая точка жизни -- 540 год до н.э. Выступал с критикой греческой мифологии, критиковал поведение Богов, писал, что Бог един. Ему принадлежит идеология пантеизма, т.е. "всё есть Бог", по учению Ксенофана в мире есть божество, как душа в человеке, оно управляет мыслью и непохоже на людей. Ещё одна мысль, что люди довольствуются только догадками о реальности, а реальная истина всёё же скрыта. 

Ксенофан повлиял на главного представителя Элейской -- Парменида(акмэ -- 504 г до н.э.), написавшего поэму <<о природе>> -- первого последовательного рационалиста. Парменид в своей поэме говорит о том, что к нему снизошло некое откровение: <<Бытие есть, небытия нет>>, ну, и <<Быть и мыслить -- это одно и то же>>, -- совпадение бытия и мышления. Парменид вводит в в философию понятие бытия; всё дело в глаголе "быть", который  в русском языке выступает как глагол-связка, он введён, чтобы отличать то, что есть независимо от наших ощущений, от того, что нам кажется: <<Глаза и уши -- плохие свидетели>> (чувства обманчивы). Реальность такой, какая она есть, открывается нам с помощью мышления. 

Рационализм -- это мировоззрение, которое доверяет только выводам разума, реальность открывется человеку только при помощи мышления. Парменид направил греческую реальность по другому пути, по пути науки. То, что кажется, -- формирует мнение, а то что есть, -- знания.

Следствия из аксиомы Парменида: небытья нет, то бытье вечно, всегда одно и то же, оно совершенно, никуда не может пропасть -- как в законах сохранения; движение -- это тоже иллюзия, с точки зрения Парменида, -- это крайний рационализм. Это привело к некой полемике. Об этом потом писал А.С. Пушкин: "<<Движенья нет>>, -- сказал мудрец; Другой же стал вокруг него ходить". Но чувства обманчивы, как и ощущение движения Солнца вокруг Земли.

Зенон Элейский, ученик Парменида, является отцом диалектики (искусства убеждения). Он предложил несколькой <<апорий>> (проход между рифами, безвыходная ситуация), родоначальников метода доказательств от противного (нельзя пройти бесконечное расстояние за конечное время, например). Другая "апория" Зенона -- про стрелу, основанная на неделимости момента времни (стрела при движении может занимать места больше, чем она есть).

После Элейской школы был Эмпедокл, по его мнению было 4 первоэлемента: вода, огонь, земля и воздух. Есть две силы: любовь и вражда, которые всё рождают и убивают, но количество частичек неизменно.

Первого из Афинских философов -- Анаксагора -- волновал вопрос метаморфозов в природе. Он хотел разобраться в процессе появления чего-то из другого, непохожего. Как потом говорил Аристотель: "Любая философия начинается с умения удивляться". Анаксагор говорил, что "Всё сложно" в том смысле, что во всё есть другие подобочастные частички, и свйоства макрообъекта отличаются количеством микрообъектов -- всё делимо до бесконечности, перераспределением этих частичек и объясняются метаморфозы. Анаксагор вводит понятие мирового ума ($\nu o \upsilon \varsigma $).

 А какую функцию он выполняет, мы узнаем на следующей лекции.

\newpage

\section*{2. Лекция о Сократе и софистах}

На прошлой лекции мы остановились на Анаксагоре и его тезисах о том, что все вещи состоят из мельчайших частиц, которые тоже состоят из мельчайших частик, и так до бесконечности. По представлениям Анаксагора, было некое однородное смешанное вещество из этих качественно неразличимых частиц. Анаксагор вводит понятие мирового космического ума ($\nu o \upsilon \varsigma $), который всё пронизывает и приводит эти мельчайшие частицы в движение, в результате которого, тяжёлые частицы начали собираться в центре, а лёгкие  -- на периферии, так родился космос -- порядок. Эта мысль потом встретится у многих философов: мировой ум как устроитель мира знает о нём всё, а человек всего лишь причастен к устройству реальности.

Анаксагор имел проблемы в Афинах, назвав Солнце <<раскалённой глыбой>> (тогда было принято, что Солнце -- это Божество). Перикл спас его от судебных преследований. 

\subsubsection*{1. Демокрит}
Последним в ряду доскоратиков считается младший современник Анаксагора -- Демокрит (470 -- 360 гг. до н.э.). Он в юности путешествовал, учился и оставил после себя более 50 произведений всякого разного рода: о природе, эстетические, сочинения об ощущении, обонянии, о числах и проч. Но в историю Демокрит вошёл в качестве основателя философского атомизма. Согласно учению Демокрита, сущее неделимо, состоит из неделимых $ \alpha \tau o \mu \varsigma $ -- $ \iota \delta \varepsilon \alpha $. Как буквы, элементарные единицы текста, отличаются формой, так и атомы, <<кирпичики>> всего сущего. Он считал, что атомы могут отличаться формой, положением и последовательностью. Демокрит тоже был рационалистом, но признавал также и $\tau o $ $\mu \eta $ $o \nu$ -- $\tau o $ $\kappa \varepsilon \nu o \nu$ (пустота).

Как и у Анаксагора, у Демокрита присутствует некоторый думализм сущего и пустоты. Как потом напишет Аристотель, Демокрит ввёл это понятие пустоты, потому что считал движение реальным,  а не иллюзорным, как выходило у Парменида. Атомы Демокрита имеют характеристики бытия Парменида, но условие их движения есть пустота.

Демокрит также оставил свой след в истории античной этики: правило жизни, искусство жизни не отдeлимы от познания и умственной деятельности. Слова -- тень деятельности, по его словам. В этике Демокрит делает центральным термин $\varepsilon o \vartheta \upsilon \mu \iota \eta$ (эвтюмия), который можно перевести, как <<хорошее расположение души>>. Он рассуждает, что такое счастье. [его современники считали счастьем удачное совпадение обстоятельств или нечто, дарованное Богами]. Демокрит же считал, что счастье и несчастье в душе; он приводил пример людей, которые завидуют богатым и боятся быть будными, но замечает, что бедный -- это не тот, которому мало, а тот, кто постоянно возмущается тем, что хочется больше. Если человек доволен тем, что у него есть, его эмоциональное состояние становится гораздо лучше. Сам же философ писал: <<Ни телесные силы, ни деньги делают людей счастливыми, а правота и многостронняя их мудрость>>. Человека отличают не телесное развитие, а благородные поступки. Демокрит подчёркивал, что человек должен стыдиться не только других, но и своей совести, выдвигая в качестве идеала человека понятие <<апатия>> (можно перевести как <<бесстрастие>>, <<бесстрадательность>>) -- человек остаётся невозмутимым к внешним раздражителям, не выходя из пределов собственной меры. <<Сильный для него тот, кто сильнее врагов, а тот, кто сильнее своих удовольствий>>, -- пишет философ в работе о храбрости. [В дальнейшем философом называли людей, умеющих преодолевать тяжёлые обстоятельства жизни, не выходя из колеи].

\subsubsection*{2. Новое время античной философии. Философия софистов}

С Демокритом заканчивается период <<Досократиков>>, наступает период 5 -- 4 вв. до н.э. Молодёжь, в том числе и сельская, устремляется в города, для того чтобы достичь каких-то успехов в жизни; меняется привычный до этого уклад. Сама жизнь в полисе, пёстрая, шумная, сделала востребованной образовательную деятельность, чтобы занимать руководящие места, выступать на собраниях.

Речь идёт об образовании, позволяющем влиять на окружающих с помощью слова. Появились софисты -- люди, которые за большие деньги предлагали следующее: в городах устраивали публичные выступления, побеждая любого человека в споре, доказывая и опровергая что угодно. Софисты считали эту науку, риторику, самым важным знанием.

Но какова их философия? 

Наиболее известный из софистов, Протагор, выдвинул философскую теорию, во многом обосновывающую движение софистов. Ему принадлежит фраза: <<Человек -- мера всех вещей>>. Если до этого искали меру как закон существования в реальности, то Протагор утверждает, что каждый сам себе мера, то есть не надо приводить себя в соответствие с чем-то другим, потому что нет что-то такого, что <<существует за мнениямии>>: когда мы ощущаем вещь, мы с ней сталкиваемся, меняя и вещь и собственные ощущения, поэтому образ этой вещи, которую мы имеем в нашых чувствах, уже устарел, потому что ничего не остаётся одним и тем же. Наши образы, слова и ощущения ничему вообще не соответствуют, каждый по-своему прав.
Отсюда, надо забоиться не о том, чтобы наши слова чему-то соответствовали, а чтобы они были сильными, чтобы они влияли на других. Слова меняют людей, по мнению софистов.

Как это отразилось в этике? Сама позиция Протагора и других софистов -- скептицизм. У каждого своё добро и зло. Но как же тогда жить в обществе? Законы общества основаны на какой-то реальности или какая-то удобная договорённость между людьми. Они, конечно, придерживались второй позиции: эти законы можно менять, можно даже им не следовать.

Данный поход, разумеется, приводил к тому, что человек терял <<опору>>, ведь ни на что полагаться более не нужно, только на самого себя. Философия Сократа возникла как раз в этом контексте.  


\subsubsection*{3. Сократ}

Сократ(469 -- 399 гг. до н.э.) же выдвигает своё учение как альтернативу софизму. Он родился и вырос в Афинах, был сыном каменотёса. Заинтересовался филсофией в юности, слушал Анаксагора. Сократ не написал никаких книг, но оставил огромное наследие в виде таких учеников, как Платон (делал Сократа героем своих произвдений), Ксенофонт(написал историю Сократа). Сократ был первым, кто спустил философию с небес на землю, привёл её в дома людей. Позиция Сократа контрастирует с позицией софистов в области различия добра и зла. Для Сократа мудрость -- умение различать добро и зло. 

Вопрос об истине. Он, как и досократики, различает истину и ложь: знание, которое отражает истину, и мнение. Он вырабатывает специальный метод получение знаний, который получил название маевтика (<<повивальное искусство>>): в отличие от софистов, он помогает найти человеку верный путь к знанию и добру. Найти такой путь можно с помощью вопросов и ответов. Сократ начал с определений, выражения сущностей. В качесте примера можно привести встречу философа со своими однополчанами, которые вели своих сыновей к софистам, чтобы обучить их воинскому делу. Завязывается разговор, ключевым вопросом которого был вопрос о том, что я является ключевым качеством для воина, -- мужество, но как его приобрести? Настоящее мужество -- не разумность без добродеятели.

Рассуждение -- это путь от мнений к знаниям, возможно, проходимый путём индукции. <<Познай самого себя ($\gamma \nu \varpi \vartheta \iota \sigma \varepsilon \delta \upsilon \tau o \nu$)>>, -- говорил Сократ. Для него тело -- это всего лишь инструмент, правильно осуществляющий желания Души. Сократ был уверен, что душа становится лучше, когда она приобретает знания и добродеятель (некие качества, которые помогают правильно относиться ко всему). Философ был уверен, что люди плохо устраивают жизнь, потому что не понимают, как правильно отличать добро от того, чем добро не является.

Сократ был уверен, что мир устроен Божеством. Глаз человека устроен, чтобы видеть, но должно быть некоторое разумное начало, которое его устроило; то же самое и с разумом: разум не может появиться и чего-то более низшего, чем он сам, -- разум дан человеку от Бога.

 Сократ закончил свою жизнь, будучи казнённым в Афинах. Демократы в Греции инициировали процесс против Сократа, который принял некие решения, которые боролись с интересами действующей власти. Этот процесс был задокументирован Платоном в его первом произведении <<Апология Сократа>>, на нём Сократ сам защищал себя на ней, поэтому мы много можем знать о его характере и верованиях. На вопрос: <<Кто самый мудрый среди эллинов?>> -- Дельфийский Оракул, назвал имя Сократа, который ответил, что он знает только то, что ничего не знает; самое сложное занятие, по мнению Сократа, -- это управлять людьми. Сократ беседовал с этими людьми (не можешь сделать своих детей достойными людьми -- какой ты управленец), поэтами (красиво говорят, но понимают ли, о чём?) и ремесленниками (разбираться в одном аспекте чего-то -- не значит считать себя умным) и понял, что они не видят меру своего незнания.
 
Критика Сократа софистами. Некий софист считал, что мужество и разум -- это инструменты для осуществления своих желаний. Сократ же ему ответил, что такой человек -- это дырявая бочка; Но если сосуд полный -- это каменная, устоявшаяся жизнь, однако больной часоткой подходит эти определения [без здорового тела невозможно удовлетворение своих желаний -- то же и с душой]. Нужно работать над душой. А то, что истины нет, -- это истина? 

\section*{4. Заканчиваем по Сократу. Философия Платона}

Сократ считает самым важным то благо, которое свяано с душой, например, знания и истина. Ими можно делиться и они умножаются, будучи равнодоступными всем. В этих благах Сократ видит путь к человеческому счстью. Наслаждение же философ считает неким дополнением к счастью. Идеал жизни -- добродеятельная разумная жизнь. Ровно так же, как человек в знании схватывает что-то универсальное, так и единый божественный разум устроил и природу -- в ней всё целесообразно.


Если подводить итог влияния Сократа для последующей философии, то он ввёл в эту науку индуктивный метод и последовательное определение, полагая, что нельзя ничего высказать о предмете, не давая его понятия, сущности предмета.
Для софистов у понятий нет объективного содержания, но задача человека перейти о мнений к знаниям. Также Сократ выдвинул на первый план проблему самопознания человека -- познания природы своей души, её естественных потребностей, целей. Для Сократа человек -- некая тайна. Поэтому и философия Сократа стала поворотным пунктом в истории античной философии -- она обратилась от познания природы обратилась в изучению человека как разумного существа.

После смерти Сократа появились т.н. сократические школы, для которых Сократ был образцом философа, мудреца. Представители этих школ на первый план выдвигали нравственные вопросы, желание быть добродеятельным. Но на этом и сходство и заканчивалось. Рассмотрим 2 такие школы.

\subsection*{1. Сократические школы}
\subsubsection*{1. Школа Киников}
Эта школа просуществовала около 600 лет. Название она получила от гимнасия Киносарг, где проходили собрания членов этой школы. Основателем этой школы является Антисфен. Он слушал беседы Сократа.

Киники особенно восприняли умение Сократа обходиться малым в жизни: <<Иные люди живут, чтобы есть, а ем, чтобы жить, >> -- говорил Сократ. Мудрец умеет извлекать пользу из любых обстоятельств. Киники говорили: <<Чем меньше человеку нужно, тем ближе он к Богам>>, и отсюда избрали свой особенный образ жизни, у них не было своего места жительства, хорошего одеяния. Но в этом они видели свободу от привязанностей. Киники создали даже целый жанр сатирической литературы, где высмеивали человеческие пороки.

Одним из известных киников был Диоген, который жил в каменной бочке. Как-то Диоген пришёл на пир, где был другой ученик Сократа, Платон, который заметил: <<Сквозь дыры твоего плаща, Диоген, светится тщеславие>>. Стремление эпатировать публику своим внешним видом и поступками было очень важным. 

Впоследствие на латыни это слово записывалось, как cynici (первая буква произносилась, как k). Вот эти их произведения, где они осмеивали все пороки человека, там не было тепла, сочувствия другим людям, это был смех такого превосходство. Настоящее совершенство же готово помочь всем подняться до своего уровня.

Киники не были на центральной дороге философии. 

\subsubsection*{2. Школа Киренаиков}
Параллельно со школой киников развивалось и движение киренаиков. Название происходит от города Кирены, расположенного в северной Африке. Основателем этой школы был Аристипп. 

Касательно нравственности киренаики считали, что цель жизни -- это счастье. Они считали, что человеку недоступно знания, -- реального знания нет, человек замкнут в знаниях своих ощущений. Значит, не надо искать объективную меру, следовательно, надо искать больше приятных ощущений и избегать неприятных; такая политика получила название гедонизма. Именно от киренаиков пошло <<Только настоящее наше>>, аналог латинского carpe diem. Они понимали это в смысле того, что благо приносит общее переживаемое чувство наслаждения. Нравственный нормы киренаики подобно софистам объявляли предрассудками.

Среди предствителей этой школы выделим Гегесия, прововедника самоубийств. Логика такая: если человек видит счастье только в получении наслаждения, то находит, что они потом приносят старадания по тем или иным причинам. Старадаешь, чтобы заполучить наслаждения, потом старадаешь в поисках новых. Если так уж тяжело быть счастливым -- цель неосуществима, то, возможно, проще совершить самоубийство.

Киренаики опробовали т.н. путь безоглядного гедонизма, неслучайно эта школа не имела продолжения.

\subsection*{2. Платон -- главный ученик Сократа}

Имя Платон -- это литературный псевдоним (буквально <<широкий, широкоплечий>>). Изначальное имя при рождении -- Аристокл. Годы жизни 427 -- 347 до н.э. Платон происходил из очень знатного рода: был потомком последнего Афинского царя, учень уважаемого, по одной линии, а по другой -- был потомком Солона, Афинского законодателя. В юности интересовался политикой, ибо его родственники, Критий и Хармит, входили в состав правящих тогда в Афинах 30 тиранов; пробовал писать стихи.

Всё изменила встреча с Сократом, и Платон решил посвятить жизнь философии. После Сократа возник даже такой жанр в греческой литературе, как диалог, когда повествование ведётся как беседа нескольких лиц. Во всех, кроме последнего, диалогах Платона главным действующим лицом является Сократ. Но отсюда сразу вопрос: где в этих диалогах заканчивается Сократ и начинаются собственные мысли Платона. 

Если говорить об учении Платона, то центральное место занимает учение об идеях ($\varepsilon \iota \delta o \varsigma$  -- <<вид>>). С точки зрения Платона, нужно различать мир становления (Гераклитовский мир, где всё течёт) и мир бытия. Мир становления -- это мир чуственно воспринимаемых вещей, а мир бытия -- мир идеи. Идеи вечные, они всегда одни и те же, они совершенны. 

Пример: корабль. Он каждый ден незаметно меняется, может и сгореть, и утонуть. Однако идея корабля: его конструкция, назначение и прочее никуда не денется.    
У Платона к миру бытия также относятся и числа.

К первому миру относятся вещи, воспринимаемые чувствами, а ко второму -- числа и идеи, воспринимаемые умом. Ум, по Платону, -- это тоже своеобразное зрение: когда человек понимает, он тоже своеобразно видит. 

Идея даёт предствление о целом, оно даёт знание, как соединять части.

В ранних диалогах Платона Сократ задаёт вопрос тем или иным собеседникам, пытаясь найти идею чего-либо. Платон впоследствие считает, что в мире бытия есть некое трансцендентное первоначало (единое, благо), которое выше его понимания. Таким образом, Платон выстраивает некую линию устройства нашего знания:

\begin{itemize}
\item Мир умопостягаемый (ЗНАНИЯ):
\begin{itemize}
\item идеи, постигаемые умом;
\item дискуссивная способность рассудка, позволяющая выводить одного из другого (пример: геометрия) -- место науки в наше время;
\end{itemize}
\item Мир чувственновоспринимаемый (МНЕНИЯ):
\begin{itemize}
\item вещи (природа) -- чувства;
\item видимость (область образов, фантазий, иллюзий) -- воображения;
\end{itemize}
\end{itemize}

Платон объяснил, что ложь возникает, когда иллюзии выдаются за реальность.

Цель человека: двигаться от мнений к истине.

Наряду с единым, с миром бытия, существует ещё и начало материи, которая даёт вещи, которые существуют временно, что изменчиво. У Платона есть образ фигурок из металла, для конструирования которых нужна идея, а материя металла она есть и так.

\subsection*{1. Антропология Платона}
Что есть человек по Платону?

Человек состоит из двух частей: душа (относится к миру идей, её нельзя увидеть, как не кромсай человека), тело (постепенно меняется, относится к чувственно воспринимаемому миру).

А если в корабле изменить все детали, будет ли этот тот же самый корабль? С точки зрения материи, нет. То же и с человеком. Человека можно узнать по душе, она может и приобретать знания, познавать мир бытия, но идея будет та же.

Душа имеет дело с миром идей: <<Подобное познаётся подобным>>, -- как говорим Эмпидокл. Она имеет три уровня (три способности):
\begin{itemize}
\item разумная (ум);
\item яростная (эмоции); 
\item вожделеющая (желания, ближе всего  к телу);
\end{itemize}
(см. диалог "государство")

Человек, по Платону, находится в праивльном состянии, когда правит разумная часть души, яростная -- союзница разумной, а не вожделеющей, а последняя -- управляемая.
Если главная вожделеющая, то человек в рабском состоянии своих жеданий.
Данную картинку можно представить в виде колесницы с двумя конями, чёрным и белым.

Душа -- бессмертна, при физической смерти просто разрывается связь между телом и душой. Душа нечто неделимое, а материя распадается.

\subsection*{2. Диалог <<Государство>>}
По аналогии со способностями души, существует три сословия:
\begin{itemize}
\item правители-мудрецы (философы -- люди, любящие мудрость) [добродеятель -- мудрость];
\item воины-стражи [добродеятель -- мужество];
\item все остальные [добродеятель -- умеренность];
\end{itemize} 

у первых двух сословий не должно быть ничего личного, они должны служить общему, служить во благо государства, их нужно специально воспитывать, готовить к службе;
у него ключевая дисциплина -- математика, самая умопостигаемая наука. 
Когда все не меняют свои системы ценностей, устанавливается справедливость.


Платон сравнивает государство по системе ценностей.

\begin{itemize}
\item Самое совершенное по Платону -- аристократия (<<власть лучших>> == власть мудрых, кого считать лучшим -- большой вопрос) [главное -- мудрость, истина].
\item Тимократия (<<власть чести>>) [главное -- доблесть, мужество]. Уже сами не ищут мудрость. Образец -- Спарта.
\item Олигархия. [главное -- богатство]. Богатство даёт независимоть, становится не важно, как ты приобрёл богатство, как ты им распоряжаешься, главное -- нести прибыль. Люди, которые ничего не делают, -- трутни по Платону, они рождают бурю в государстве.
\item Демократия (каждый живёт как хочет -- вожделеющая чать души активная, человек становится неустойчивым) [главное -- свобода]. Возникают явления, что учителя боятся учеников -- притесняется демократия, всё принудительное -- возмутимо. Демократия создаёт почву для Тирании.
\item Тирания (полностью доминирются вожделеющей частью души, а яростное начало -- его союзник. 
Согласно Платону, демократический строй формирует собой предпосылки для ещё более деградирующего общественного устройства -- тирании. Это связано с тем, что в демократическом обществе формируется демократический человек, который балансирует между своими желаниями --  всё, что является желанием, достойно осуществления. Прослойка людей, порабощённых своими желаниями, готовыми на всё, формирует базис тирании.

Сравнивая человеческую душу с Аркополем, Платон пишет о процессе формирования тиранического человека; если человек попадает в общество похожих на него, особенно молодых порочных людей, внутри человека начинвется борьба. 
Душевное богатство умений и знаний подобно Акрополю, а такая компания -- штурм этого Акрополя. Если этот Акрополь оказывает недостаточно защищённым, то вожделения вносят в душу человека <<ложные мнения, хвастливые речи>>. Внутри человека происходит изменеия взглядов на всё. Отрицательное, например, наглость и безрассудство начинает восприниматься как геройство. 

\end{itemize}

Ещё одна тема, затронутая в диалоге, -- тема значения искусства для общественной жизни, где его социальная философия переплетается с эстетикой. 

Он исходит от распространённой античной концепции: искусство возникает от подражания. Платона задаётся вопросом, чему подражают поэты и драматурги, когда изображают человека. Размышляя над этой проблемой, он приходит к выводу, что поэты, не будучи сами приверженцами размной стороны, изображают в человеке его неразумную сторону, то есть его яростное и вожделеющее начала -- такие герои имеют успех у толпы. Философ полагает, что когда поэт обращается к такому яростному и переменчивому нраву, то это нравится зрителям, хорошо поддаётся воспроизвдению; но тем самым поэты питают и укрепляют эту часть души человека, никак не питая разумную часть, и этим самым дают зрителям образцы для подражания. Платон относится к [поэтическому] воображению очень осторожно и даже отрицательно, что характерно для античной философии, потому чтто воображаемые предметы не обладают полноценным бытием: воображение коснструирует иллюзорные, обманчивые образы.  То же и применимо по отношению к музыке.

\section*{5. Аристотель -- ученик Платона}

Аристотель (384 -- 322 гг. до н.э.) был родом из Македонии, провинциальной части северной Греции, происходил из семьи придворного врача.
О нём Платон писал: <<Если для некоторых учеников нужны шпоры, чтобы заставлять их учиться, то для Аристотеля нужна узда>>. Аристотель стал образцом учёного. 

Первые произведения Аристотеля (диалоги, подобные Платону, и какие-то произведения) не дошли до нас, зато дошли конспекты лекций, которые он читал в своей школе. 

Аристотель около 20 лет учился в академии Платона, занятия там были неким образом жизни: были и учёба, и исследования. После учёбы Аристотель по приглашению Македонского царя Филиппа едет быть учителем для его сына, Александра. Однако учёба была не долгой -- Филиппа убивают. После этого философ возвращается в Афины и созадёт там собственную философскую школу <<Ликей>> в 335 г. до н.э. Ещё другое название этой  школы -- перипатетики -- <<прогулка с беседой>>. Его интересовали различные разделы науки от физики до социальной философии. Потом после смерти Александра в Афинах вспыхивает восстание, Аристотетль вынужден был бежать. 

\subsection*{1. Классификация знания по Аристотелю}

Согласно Аристотелю, опыт отличается от знания тем, что знающий отвечат на вопрос: <<что и почему?>> -- то есть знает причины тех или иных событий	и способен эти знания передать другому.

Виды знаний [по важности]:
\begin{itemize}
\item теоретические (<<обозревать>>) знания (науки): первая философия (изучает сущее в целом, первопричины всего), вторая философия -- физика.
\item практические (<<действовать>>) науки: этика, политика, экономика (законы ведения домашнего хозяйства)
\item пойетические (<<творчество>>)
\item Логика по Аристотелю -- законы мышления и доказательства
\end{itemize}

\begin{table}[]
\begin{tabular}{lll}
\textbf{Виды знания (науки)} & \textbf{Предмет}  & \textbf{Цель}                                                           \\
\textit{Теоретические}     & сущее (независимо от нас)   & ради знания                                                    \\
\textit{Практические}       & деятельность {[}человека{]} & \begin{tabular}[c]{@{}l@{}}правила для\\ действий\end{tabular} \\
\textit{Пойетические {[}творческие{]}} &
  \begin{tabular}[c]{@{}l@{}}создание того, чего нет\\ в природе\end{tabular} &
  \begin{tabular}[c]{@{}l@{}}средство для \\ деятельности\end{tabular}
\end{tabular}
\end{table}


\subsection*{2. Онтология}

Аристотель, анализируя понятие сущего -- что есть, приходит к выводу, что мы используем это понятие в разных значениях. Он вводит в обиход понятие категории -- базиса, с помощью которого мы описываем и классифицируем сущее.	

Аристотель относит к сущему:
\begin{itemize}
\item[1.] сущность [отвечает на вопрос <<что?>>], например, человека,
\item[2.] качество [отвечает на вопрос <<какой?>>], например, молодой человек,
\item[3.] количество [отвечает на вопрос <<сколько?>> и прочие], например, рост этого человека,
\item[4.] отношение [отвечает на вопрос о взаимосвзяи], например, брат, ученик и проч.,
\item[5.] место [отвечает на вопрос <<где?>>], например, в Греции,
\item[6.] время [отвечает на вопрос <<когда?>>],
\item[7.] положение в пространстве,
\item[8.] обладание/лишённость, например, вооружённость,
\item[9.] действие,
\item[10.] претерпевание [пассивный или активный залог действия].
\end{itemize}


Первая категория -- самая главное, остальное -- её характеристики.

Сущность есть первая и вторая: вторая сущность -- человек, а первая -- конкретное существо. Первая сущность -- подлежащее в предложениях, не являясь характеристикой чего-то другого. Вторая сущность -- идея по Аристотелю. Если по Платону идея -- это нечто полоноценное, существующее независимо от чего-то другого, а вещи -- тени идей. А у Аристотеля вторые сущности не существуют без первых. 

А как описать первую сущность без второй? Аристотель вводит определение: первая сущность == материя (некий субстрат) + форма(идея, вторая сущность); пример медного шара: существует материальный шар, только один, отдельно от математической абстрации шара он не существует.
Чтобы появился такой шар, должна быть:
\begin{itemize}
\item материальная причина (из чего),
\item формальная причина (что?),
\item движущая (от чего? что приводит к этому?), 
\item целевая (ради чего?) [Природа не похожа на трагедию].
\end{itemize}

Ещё одно определение материи по Аристотелю -- $\delta \upsilon \nu \alpha \mu \iota \varsigma$ (от него -- <<динамика>>, сила, возможность [чем-то быть]). Чтобы что-то реализовалость (первую сущность), нужна $	\varepsilon \nu \varepsilon \rho \gamma \varepsilon \iota \alpha$ (<<действительность>>). Любая вещь действительно существует только тогда, когда она действительно функционирует в соответствии со своей природой.

Аристотель ставит вопрос: есть ли такая действительность, где нет материи, одна форма? По Аристотелю существует мировой божественный разум, который постоянно мыслит, занимаясь чистой разумительной (высшей по Аристотелю) деятетльность. Божественный разум -- форма форм. Без божественного разума по Аристотелю не могла бы существовать природа. Действительность первее возможности.

\subsection*{3. Этика}

Когда речь идёт о человеке, его совершенство проявляется в его деятельности. Человек, по Аристотелю, $\zeta \omega o \nu$ $\lambda o \iota \gamma o \nu$ $\kappa  \alpha \iota$ $\pi o \lambda \iota \tau \iota \kappa o \iota $ : <<Живое существо разумное и социальное>>. 

Этика Аристотеля продолжает традиции Сократаи и Платона, считая, что цель деятельности -- это благо. Виды благ по Аристотелю:
\begin{itemize}
\item внешние (мы ими можем обладать): богатство, власть, почёт;
\item телесные: здоровье и чувственная красота;
\item душевные: добродеятели Сократа -- самое главное по Аристотелю:
\begin{itemize}
\item разумная часть: мудрость, хорошая память, благоразумие и т.д. [путём обучения]
\item неразумная часть: наш нрав -- этос, наши привычки [путём упражнения].
\end{itemize}
\end{itemize}

\end{document}